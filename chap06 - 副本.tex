%!TEX root = D:\thesis\thesis_bx.tex
\chapter{轻量级CoAP协议的安全性分析及改进}
华为NB-IoT 战略中的首款芯片 Boudica 120 以及 LiteOS
操作系统支持CoAP协议。


\section{CoAP协议介绍}
CoAP,即约束应用协议,与HTTP协议一样也是
一种文档传输协议。而不同的是,CoAP是针对受限
设备的需求来设计的,其数据包大小远小于HTTP的
TCP流。CoAP通过 UDP协议运行,客户端和服务器
通过无连接数据包进行通信,并且允许UDP广播和多
播用于寻址。其结构如图\ref{fig-coap}
\begin{figure}[htp]
	\centering
	\includegraphics[width=13cm]{data/content/chap05/CoAP协议}
	\caption{CoAP结构图}
	\label{fig-coap}
\end{figure}
CoAP协议遵循客户端/服务器模型,客户端向服务
器发出请求,服务器对请求发回响应。模型中的客户
端和服务器是功能逻辑上的划分,节点可兼具客户端和
服务器的功能[5]。像 HTTP一样,CoAP基于非常成功
的REST模型:服务器使资源在URL下可用,客户端使
用GET,PUT,POST和DELETE等方法访问这些资源。

CoAP协议的开源实现,有三个版本较为实用:

(1)libcoap
ibcoap是针对资源受限(如计算能力、射频范围、
内存、带宽、网络数据包大小等)设备的轻量级应用程
序协议的C实现。该协议由IETF作为RFC 7252标准

(2)Californium
基于 Java 实现的 Californium 是另一个很实用的
开源版本。Californium 是一个强大的 CoAP 框架,它
为技术人员开发与调试提供了很方便的API。

(3)node-coap
node-coap 是建立在 Node.js 的 http 库上的一个
CoAP协议客户端和服务器的开源库。它也是一种很
流行 CoAP 实现方式,具有快速搭建、简单配置的特
性,可以方便地使用它进行测试与开发。

以 CoAP实现,类似于传统的Web服务器,单车以
GET、POST 等方式向服务器发送请求,例如认证、故
障等,服务器对请求作出响应,单车使用返回的响应
信息进行相应的操作。此时,服务器需要识别每一个节点,信息交互需要二者同时处于在线状态。


\section{CoAP协议模型}


\subsection*{协议模型}

\subsection*{安全属性建模}
对CoAP协议进行建模,其模型可以看作

\section{改进的CoAP协议}


\section{实验与结果分析}