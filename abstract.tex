


\begin{cabstract}

协议的安全性分析一直是网络安全等领域的重点研究问题。近年来,随着智能终端、云计算、5G技术等新兴技术的快速发展,以移动互联网、物联网、卫星互联网等
为代表的新一代网络方兴未艾。在这些新一代网络中,协议作为构建网络的必要组成部分,是保障整个系统的正常工作的基础,国内外研究者通过设计安全协议等方式来实现网络安全。安全协议就是通过密码学的方法为网络节点提供安全保证的分布式算法,它是当前社会信息网络安全的基础技术之一,旨在保护公共网络上的通信。安全协议对错误十分敏感,一旦存在安全缺陷,即安全协议没有满足它声明的安全性,将会对信息安全产生极大威胁,可能会造成重大的经济、社会甚至军事影响。

形式化方法是基于严格数学基础,对目标系统进行规约、开发和验证的技术。形式化方法在分析高安全性需求的关键系统时,具有不可比拟的重要性,尤其在安全协议分析领域中更加突出。从安全协议的设计到实现再到应用的过程上看,安全协议的内涵包括规范(Specification)、实现(Implemention)以及配置(Configuration)三个层次,协议的安全漏洞主要出现在规范层和实现层中。一方面,传统形式化方法中的敌手模型不足以表征敌手的全部攻击能力;另一方面,目前针对实现层的分析多聚焦在传统软件安全性分析层面,很难找到其中存在的逻辑漏洞。本文研究形式化方法对协议的规范层和实现层的安全性分析技术,以解决传统方法在协议的规范层和实现层的安全性分析中存在的不足。论文的主要研究工作及创新点归纳如下:


%%%论文围绕形式化分析技术对协议实现的安全性分析展开研究,针对符号方法中敌手描述能力不足的问题,拓展了传统Dolev-Yao敌手模型。针对安全协议实现状态机学习中的测试序列生成问题,提出了改进的测试序列生成算法。为了验证协议实现状态机的安全属性,提出了基于模型检测的协议实现分析技术。但是安全性已经被形式化技术证明了的协议,也频繁出现安全漏洞。如何通过形式化技术对安全协议进行全面和系统的安全性分析,是现实中亟需解决的问题。


%针对常规的符号执行方法无法处理多数据包输入的问题,从提高测试深度出发,提出了一种多数据包选择符号执行方法。
%提出了两类递归协议的安全性判定算法。
%该算法能够通过完备的敌手知识的符号表示,对执行正常输入输出以及递归测试的安全协议,通过约束系统建模敌手知识,并加入成员查询的递归语言,同时获得了敌手知识的完备表示。在被动敌手的情况下,给出了基于连接的递归语言和基于映射的递归语言的判定过程,并证明这是NP问题。

%将选择符号执行方法用于协议软件测试,提出了一种数据包选择符号执行方法。随后,在数据包选择符号执行方法的基础上进行改进,使之可以处理多次数据包的输入,提出了一种多数据包选择符号执行方法,提高了协议软件安全性分析的测试深度。

%针对常规的符号执行方法无法快速到达指定协议状态的问题,从提高测试效率出发,
%(1)针对符号方法中敌手描述能力不足的问题,提出了基于操作语义的敌手拓展模型。该方法结合计算方法假设敌手具有强能力以及符号方法能实现自动分析的优势,克服现有Dolev-Yao敌手模型没有贴近现实模拟诚实用户与被攻陷用户的缺点,在分解敌手攻击能力域为数据泄漏的数据类型、身份标识和发生时间的基础上建立敌手攻击规则,提高了协议形式化分析中敌手模型的精确性和实用性,能够验证协议的完美前向保密性、密钥独立性、抗密钥泄露模仿性等强安全属性。通过拓展敌手推导规则,增强敌手攻击能力,来分析传统Dolev-Yao敌手模型中认为安全的协议。通过形式化工具Scyther对HMQV协议的验证结果表明,该拓展模型能够找到传统敌手模型验证为安全的协议中的协议漏洞。

(1)提出了基于操作语义的敌手模型拓展,该模型能够表达比传统敌手模型更强大的攻击能力,找到传统Dolev-Yao敌手模型中认为安全的协议的安全漏洞。操作语义可以将敌手攻击能力在相关协议模型中形式化表示,有利于进一步丰富对敌手攻击的描述能力,拓展了敌手攻击规则,建立了敌手长期密钥攻击模型以及敌手短期数据攻击模型。通过形式化工具Scyther对HMQV协议验证,验证结果表明,该模型能找到传统Dolev-Yao敌手模型中认为安全的HMQV协议的安全漏洞,在敌手短期数据攻击模型下,HMQV协议的会话密钥是能够被敌手获得的,证明该协议在拓展的敌手模型下是不安全的。

(2)针对递归协议的递归原语难以形式化描述,导致递归协议安全性难以判定的问题,提出了基于拓展敌手模型的递归协议安全性判定算法,该算法能够对执行正常输入输出以及递归测试的协议进行验证,尤其是安全路由协议以及公钥证书路径的验证。通过对递归协议消息项、敌手能力建模,建立递归协议安全性判定约束系统。在拓展的敌手模型条件下,通过简化规则,将协议拓展敌手模型下的可推理项置于约束系统下。给出了基于映射的递归语言的判定过程,并证明这是一个NP问题。

%提出了一种基于模型的导向符号执行方法。该方法利用协议软件的交互特性和已知的协议规范信息,首先,对协议软件进行协议状态机模型推断,提出了一种针对协议软件的高效状态机学习算法。然后,利用协议模型对协议软件进行导向性符号执行,使得符号执行可以快速到达需要测试的协议状态,缓解了符号执行的状态空间爆炸问题,提高了协议软件安全性分析的测试效率。

%针对常规的符号执行方法无法联合测协议软件的接收端与发送端的问题,从发现互操作性方面的安全问题及缩减状态空间的角度出发
(3)提出了基于状态机学习的安全协议实现分析技术。基于状态机的协议实现模型提取技术存在测试序列生成慢以及测测试序列生成树状态爆炸问题,针对测试序列生成慢的问题,提出了检查点算法,通过构造前缀树用来存储之前执行过的查询,使得测试序列生成所用步骤减少,减少了状态机生成时间。针对测试序列生成数量爆炸问题,提出了套接字约简方法,根据安全协议自身的安全机制,利用影响协议后续交互的套接字,约简相应生成的测试序列。提出了安全协议实现的状态机学习框架,使用改进算法生成测试序列,利用L*算法生成状态机模型,对TLS协议实现的状态机进行学习,成功学习到了GnuTLS,OpenSSL,NSS三种TLS协议实现的状态机模型。实验证明,与传统方法相比,该技术能够显著降低了学习过程的测试序列生成步骤、等价查询数量及所用学习时间。


%针对符号执行方法需要根据协议规范人工构造输入的问题,从提高自动化程度的角度出发
(4)提出了基于模型检测的协议实现分析技术。通过对协议实现库的有限状态机提取,提取TLS协议实现库的状态机模型,建立协议安全属性模型,寻找协议实现中可能存在的异常行为,实现对协议实现库的自动化和系统化的分析。本文对利用测试用例生成的协议实现库状态机进行安全属性建模,利用NuSMV工具,对提取的模型进行模型检测,实验证明提出的方法能够有效分析TLS协议实现库的状态机模型,找到协议实现库存在的逻辑漏洞及与规范不一致的缺陷。建立了基于模型检测的协议实现状态机模型分析框架。该框架以协议实现的状态机为输入,将协议规范及实现库声明的安全属性建模为线性时态逻辑,通过模型检测器NuSMV对安全属性进行验证,将验证结果分为属性满足和不满足。同时基于协议实现声明,建立了安全属性、密钥交换属性以及函数属性3类共12项安全属性,对验证过程中产生反例通过伪反例检查和反例重放两步组成的反例检查器进行筛选,确保该反例为真实执行。通过对TLS实现库的验证结果,找到了两种TLS协议实现的协议逻辑漏洞的真实执行,证明该方法所得的模型检测结果可以发现协议规范与协议实现之间的差异,有效发现协议实现状态机的逻辑漏洞。

%针对符号执行方法需要根据协议规范人工解析协议格式,从补充协议软件符号执行技术、发现更多安全性问题的角度出发,提出了一种基于流量分析和逆向工程的协议软件模糊测试方法。首先,分析了常规模糊测试方法存在的问题。然后,分别研究了基于模块化协议描述语言的流量分析方法、基于遗传算法模型的逆向工程方法,并将两个方法有机结合起来。实验结果表面,该方法提高了模糊测试的代码覆盖率,提高了协议软件安全性分析的测试效率,对于符号执行方法中约束不可求解的情况是一个很好的补充。

\end{cabstract}
\ckeywords{安全协议; TLS;形式化方法;状态机学习}

\begin{eabstract}
The security analysis of the protocol has always been a key research issue in the field of network security. In recent years, with the rapid development of emerging technologies such as smart terminals, cloud computing, and 5G technology, a new generation of networks represented by mobile Internet, Internet of Things, satellite Internet, etc. has been in the ascendant. In these new-generation networks, the protocol, as a necessary part of constructing the network, is the basis for ensuring the normal operation of the entire system. Researchers at home and abroad implement network security by designing security protocols and other methods. A security protocol is a distributed algorithm that provides security guarantees for network nodes through cryptographic methods. It is one of the basic technologies of current social information network security and aims to protect communications on public networks. Security protocols are very sensitive to errors. Once there is a security defect, that is, the security protocol does not meet the security declared by it, it will pose a great threat to information security and may cause significant economic, social and even military impacts.

The formal method is a technique based on mathematics to describe the nature of the system. The formal method can comprehensively describe, develop and verify the target system. The typical formal method is given in a formal specification language. Formal methods are of incomparable importance when analyzing critical systems with high security requirements, especially in the field of security protocol analysis. From the design, implementation, and application process of the security protocol, the connotation of the security protocol includes three levels: specification, implementation and configuration. The security vulnerabilities of the protocol mainly appear in the specification layer and the implementation layer. On the one hand, the adversary model in traditional formal methods is not sufficient to characterize all the adversary's attack capabilities; on the other hand, the current implementation layer analysis focuses on the traditional software security analysis level, and it is difficult to find the logical loopholes in it. This paper studies the security analysis technology of the formal method to the specification layer and implementation layer of the protocol to solve the shortcomings of the traditional method in the security analysis of the specification layer and implementation layer of the protocol.

This thesis focuses on Research on Techniques for Security Analysis of  Protocols with Formal Methods. The main research work and innovative results are summarized as follows:


Aiming at the problem of insufficient description capabilities of adversary models in traditional formal methods, an adversary model based on operational semantics was established.An extended adversary model based on operational semantics is proposed. This model expands the traditional adversary attack ability, decomposes the cryptographic security attributes into three dimensions, and establishes the adversary corresponding to the data type, identity and time of occurrence of the data leakage rule. The corresponding adversary model is established through the combination of rules, and the adversary model and security properties are proved. By expanding the adversary's deduction rules and enhancing the ability of the adversary to attack, the protocol considered safe in the traditional Dolev-Yao adversary model is analyzed. Finally, HMQV protocol is verified by analysis tool Scyther. It is verified that the HMQV protocol is no longer safe under the extended adversary model.

Aiming at the problem that the recursive primitives of the recursive protocol are difficult to formally describe, and the security of the recursive protocol is difficult to determine, a recursive protocol security determination algorithm based on the extended adversary model is proposed. This algorithm can perform normal input and output and recursive testing of the protocol. Verification, especially the verification of secure routing protocols and public key certification paths. By modeling the recursive protocol message items and adversary capabilities, a recursive protocol security judgment constraint system is established. Under the condition of the extended adversary model, by simplifying the rules, the reasonable items under the extended adversary model of the protocol are placed under the constraint system. The judgment process of the recursive language based on mapping is given, and it is proved that this is an NP problem.

Aiming at the problem that the analysis model of the protocol implementation is difficult to extract, a security protocol implementation analysis technology based on state machine learning is proposed. The state machine-based protocol implementation model extraction technology has the problem of slow test sequence generation and the explosion of test sequence spanning tree state. Aiming at the problem of slow test sequence generation, a checkpoint algorithm is proposed to store previously executed queries by constructing a prefix tree , Which reduces the steps used in test sequence generation and reduces the state machine generation time. Aiming at the explosion of the number of test sequences generated, a socket reduction method is proposed. According to the security mechanism of the security protocol, the sockets that affect the subsequent interaction of the protocol are used to reduce the corresponding generated test sequences. A state machine learning framework for the implementation of the security protocol is proposed, the improved algorithm is used to generate test sequences, the L* algorithm is used to generate the state machine model, the state machine implemented by the TLS protocol is learned, and three TLS protocol implementations are successfully learned: GnuTLS, OpenSSL, and NSS State machine model. Experiments show that compared with traditional methods, this technology can significantly reduce the test sequence generation steps, the number of equivalent queries and the learning time used in the learning process.

Aiming at the problem of inability to analyze the logic loopholes of the protocol realization state machine, a protocol realization analysis technology based on model checking is proposed. By extracting the finite state machine of the protocol realization library, the state machine model of the TLS protocol realization library is extracted, and the protocol security attributes are established. Model, look for possible abnormal behaviors in the protocol implementation, and realize the automation and systematic analysis of the protocol implementation library. This paper models the security attributes of the protocol realization library state machine generated by the test cases, and uses the NuSMV tool to perform model check on the extracted model. The experiment proves that this method can effectively analyze the state machine model of the TLS protocol realization library and find the protocol to realize the inventory. In the logic loopholes and inconsistent flaws with the norms. Established a protocol based on model checking to realize the state machine model analysis framework. The framework takes the state machine implemented by the protocol as input, models the protocol specification and the security attributes declared by the implementation library as linear temporal logic, verifies the security attributes through the model detector NuSMV, and divides the verification results into attribute satisfaction and dissatisfaction . At the same time, based on the protocol implementation statement, a total of 12 security attributes are established in three categories: security attributes, key exchange attributes, and function attributes. The counterexamples generated during the verification process are screened by a counterexample checker composed of two steps: false counterexample check and counterexample replay. Ensure that the counterexample is truly executed. Through the verification results of the TLS implementation library, the actual implementation of the protocol logic vulnerabilities implemented by the two TLS protocols has been found, which proves that the model detection results obtained by this method can find the difference between the protocol specification and the protocol implementation, and effectively discover the protocol implementation state machine Logic loopholes.

\end{eabstract}
\ekeywords{security protocol; TLS; formal methods;state machine learning}
