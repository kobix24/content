%!TEX root = C:\Users\kobix\Documents\gradu\thesis1\thesiscc.tex
\chapter{基于状态机学习的协议实现模型提取技术研究}
\label{chap-statemachine}
许多在规范层面被形式化方法证明安全的协议,可能存在实现层面的漏洞。协议实现是在协议规范的基础上开发的软件或系统,这是因为协议被具体实现时,由于规范的描述不明确、编程人员的实现方式选择各异,造成了协议软件的具体实现与规范的不一致,甚至产生安全漏洞。由于协议实现的复杂性,对其进行安全性分析、发现实现层面的漏洞非常困难,很多漏洞是经过复杂的数据交互之后才暴露出来的,大大增加了分析难度。如果能够提取协议实现对应的抽象模型,则可以针对该模型进行系统分析,得到全面的分析结果。状态机学习技术能够针对协议的真实执行提取协议实现的状态机模型,并在此基础上对协议进行深入分析。

本章研究了如何提取协议实现的抽象模型,以通过形式化方法对其模型进行分析,提出了基于状态机学习的安全协议实现模型提取技术,改进了传统状态机学习算法,解决了当前在状态机学习中需要解决状态机学习时间随目标系统状态数增长而呈指数级增长的问题,显著提高了状态机学习的效率。通过改进算法对TLS协议实现的状态机进行学习,能够提取协议实现的状态机模型以分析其安全性。测试结果表明,提出的改进算法能够大幅减少状态机学习过程生成的等价查询数量,加速状态机学习过程。同时通过学习到的状态机模型,分析其异常状态,找到一个OpenSSl的逻辑错误,证明学习到的模型是有效的。


\section{状态机学习简介}

状态机学习也称状态机提取,是通过分析对象的实际执行中,根据状态迁移条件,获取运行时的状态转换信息,从而逐步构造分析对象的状态机的技术。通过状态机学习技术,可以建立安全协议实现对应的状态机模型。通过对协议状态机的分析,能够将协议实现的安全属性分析问题转化为该状态机的安全属性分析,这有助于对协议实现的状态状态转移关系进行细致的分析,找寻其中对应的逻辑问题,而关注点将不再是常规程序分析中对协议实现的内存问题的寻找。本节主要介绍状态机学习算法中常用的米利状态机及观测标构造算法。

\subsection{状态机学习算法}


通常协议状态机的状态转移不仅与当前接收到消息有关,也与此时状态机的状态有关。如处在等待ServerHello状态时的客户端,会接受并解析目标服务器发来的ServerHello消息,并准备进入握手状态;而若已经在握手阶段,则不会根据收到的ServerHello包来转移到握手状态。因此,协议状态机的输出,与当前接收的输入及所处的状态有关。
本章选择Mealy状态机来进行协议状态机学习。



\begin{definition}{(米利状态机)}

米利(Mealy)状态机是一类有限状态机,可以形式化地描述为一个六元组\[M=<I,O,Q,\delta ,\lambda ,{{q}_{0}}>\] ,其中:
 $I$是一组有限的输入集;
 $O$是一组有限的输出集;
 $Q$是一组有限的状态集;
${{q}_{0}}\in Q$是初始状态;
 $\delta :Q\times I\to Q$是迁移函数;
 $\lambda :Q\times I\to O$是输出函数。

根据定义,输出函数$\lambda $满足
$\lambda (q,\varepsilon )=\varepsilon $,$\lambda (q,i\sigma )=\lambda (q,i)\lambda \left( \delta \left( q,i \right),\sigma  \right)$
%
其输出依赖于状态机当前状态和输入。米利状态机的行为可以通过函数$\mathsf{}{{A}_{\mathsf{\mathcal{M}}}}:I*\to O*$来定义,其中${{A}_{M}}(\sigma )=\lambda ({{q}^{0}},\sigma )$。状态机$\mathsf{\mathcal{M}}$和$\mathsf{\mathcal{N}}$称为等价的当且仅当\[{{A}_{\mathsf{\mathcal{M}}}}={{A}_{\mathsf{\mathcal{N}}}}\],记做$\mathsf{\mathcal{M}}\approx \mathsf{\mathcal{N}}$。
\end{definition}
%

%




%



%




\begin{example}
一个Mealy状态机实例如图\ref{example-mealy}所示,状态$q_{a}$为初始状态,
有向箭头表示状态迁移关系,箭头标注了输入输入符号。
为描述方便,该实例中用自然数表示输入输出符号。
\end{example}

\begin{figure}
	\begin{center}
		\begin{tikzpicture}[->,>=stealth',shorten >=1pt,auto,node distance=5cm,thick]
		\tikzstyle{every state}=[fill=blue,inner sep=.30cm,draw=none,text=white]

		\node[initial,state] (A)                   {$q_a$};
		\node[state]         (B) [below left of=A] {$q_b$};
		\node[state]         (C) [below of=A]      {$q_c$};

		\path (A) edge [loop above]   node {$1/2$} (A)
		edge    node {$2/3$} (B)
		edge [bend left]    node {$3/1$} (C)
		(B) edge [loop left]    node {$1/3$} (B)
		edge [bend left]    node {$3/2$}  (A)
		edge    node {$2/1$} (C)
		(C) edge [loop below]    node {$1/1$} (C)
		edge [bend left]   node {$3/3$} (B)
		edge              node {$2/2$} (A);
		\end{tikzpicture}
		\caption{Mealy状态机实例}
		\label{example-mealy}
	\end{center}
\end{figure}




在Mealy状态机中,输入消息集合$\Sigma_{I}$和输出消息集合$\Sigma_{O}$是抽象的字母集合,
因此需要一个确定的抽象函数,将实际的输入消息和输出消息映射到相应的字母集合中。


确定性Mealy状态机的执行路径可以通过一个输入消息序列
$m_1m_2 \cdots m_n$表示,并且该消息序列满足下列条件:
$\forall i\in [1,n]$,$\exists (q, m, q')\in \delta$,
$\exists (q, m, o)\in \lambda $,
使得$m = m_i$,并且当$i=0$时,$q = q_0$。
直观上,一个输入消息序列对应了一条状态机执行路径,
并将唯一决定状态机的最终执行状态。
下文中,将不区分状态机的执行路径和输入消息序列,
因为二者在状态空间的可达性分析中是等价的。




状态机学习算法中,状态机学习算法与状态数的数量相关,数量越多,状态机学习过程越复杂。状态机学习算法就是查询所有状态,合并等价的状态,拆分存在冲突的状态。状态机学习算法中最经典的就是L*算法,很多最新的状态机学习都是建立在L*算法的基础上。

%\textbf{L*算法}



%L*算法的主要用于学习给定正则语言L的最小化DFA,包括了Learner、Teacher、Oracle三种角色和成员确认查询、等价确认查询。Learner主要通过学习获得状态机的信息, Teacher主要回答Learner的成员确认查询,Oracle用于响应等价确认查询,即判断该状态机能否识别正则语言L。代表语言L是不是接受对当前行列所组成的语句。依据这三种角色,构建一个监督表(Observation Tables),表中存储数据集$(S,E,T)$。其中$S$、$E$分别记录了有限符号表$\Sigma$中的前缀闭合集和后缀闭合集。$T(s,e)$由$S$到$E$的一个映射,令$s\in S \cup S \bullet \sigma$,$e \in E$,$T(s,e)=1$表示状态机接收$s \bullet e$,$T(s,e)$为$0$表示拒绝,“$\bullet$”是连接运算符。根据实际状态机运行
%Teacher 响应为match或反例C:$C=\{\forall t\in S | t\in U/L\_m\vee t\in L\_m/U  \}$,
%此时将C及他的前缀加入S中。
%将此状态机作为等价查询发给Teacher,Teacher比对正确的规范,将会响应match或给出反例,若给出反例,那么将该反例C及其前缀加入表中状态集S并重复该算法,若观测表为连续且封闭的,则可以定义相应的状态机。




%\end{enumerate}


L*算法可以用于学习目标系统的有限状态机(Finite State Machine,FSM),在学习算法中,包括了学习者、指导者、预言机三种角色和成员查询、等价查询两种查询。初始状态下,学习者的初始知识为米利状态机$M$的输入字母表$I$以及输出字母表$O$。指导者的知识为整个状态机,学习者通过两类查询来学习状态机模型。

成员查询:这类查询中,学习者查询一个字符串$s$以及它的输出为其是否被目标系统接受,即$\sigma \in I*$,指导者通过${{A}_{\mathsf{\mathcal{M}}}}(\sigma )$中的输出序列响应。

等价查询:这类查询中,学习者查询一个假设的米利状态机$\mathsf{\mathcal{H}}$是否正确,即是否$\mathsf{\mathcal{H}}\approx \mathsf{\mathcal{M}}$。如果正确,预言机回答为肯定;如果不正确,它会给出一个反例$C$,反例$C$是引起两个状态机不同输出序列的消息$\sigma \in I*$,它满足${{A}_{\mathsf{\mathcal{H}}}}(\sigma )\ne {{A}_{\mathsf{\mathcal{M}}}}(\sigma )$。

学习者向指导者提出查询获得状态机的信息,指导者回答学习者的成员查询,预言机用于响应等价确认查询,即判断该状态机能否正确代表协议规范中的状态机。$L^{*}$学习算法在构造状态机模型的过程中,创建并维护一个特殊的数据结构,
称作观测表,是$L^{*}$算法的核心。观测表记录了协议实现对不同输入消息的响应和输出。

通过学习者、指导者和预言机可以构建一个观测表$<S,Exp,Ot>$ 其中$S$是状态标记集,$Exp$为测试集,$Ot$为表坐标的真值映射。学习者获得信息提出假设的状态机,并将此状态机作为等价查询发给指导者,指导者比对正确的规范,将会响应符合或给出反例,若给出反例,那么将该反例$C$及其前缀加入表中状态集$S$并重复该算法,若观测表为连续且封闭的,则可以定义相应的状态机。


成员查询算法会更新观测表,观测表必须满足封闭性和一致性原则。

%\begin{enumerate}[(1)]
(1)封闭性原则:选取$S \bullet I$集合中任意一个序列$t$,一定存在一个序列$s \in S$使得$row(s)=row(t)$,即序列$s$、$t$在集合$T$中相对应的元素是等效的。

(2)一致性原则:若有$s_1,s_2 \in S$且$row(s_1)=row(s_2)$,那么对任意的符号序列$a \in I$,一定有$row(s_1 \bullet a)=row(s_2 \bullet a)$。
通过不断的成员查询,得到成员确认,最后学习出协议的状态机。状态机学习算法如图\ref{fig-chap02-state_learn}所示。

\begin{figure}[htp]
	\centering
	\includegraphics[width=11cm]{data/content/chap02/state_learn}
	\caption{状态机学习算法}
	\label{fig-chap02-state_learn}
\end{figure}

\begin{definition}{(观测表)}

观测表能够通过语句的前缀后缀结合,构造等价查询。观测表可以形式化描述为一个三元组$(S, E, T)$, 其中,
\begin{enumerate}
\item $S$是语言$L$能够接受的语句的所有前缀集合,每个字符串是$\Sigma_{I}^{*}$的元素;
\item $E$是语言$L$能够接受的语句的所有后缀集合,每个字符串是$\Sigma_{I}^{+}$的元素;
\item $T: (S \cup S\cdot \Sigma_{I}) \times E \rightarrow \Sigma_{O}^{+})$,表示语言$L$是不是接受当前行列所组成的语句。
\end{enumerate}
\end{definition}


观测表中映射$T$满足以下属性:$\forall s \in S \cup S \cdot \Sigma_{I}$,
$r\in E, x\in \Sigma_{0}^{+}$, $(T(s, r) = x) \implies |r| = |x|$。


\begin{example}
图\ref{example-mealy}Mealy状态机对应的观测表如表\ref{table-1}所示。
直观上,$S$可以看做集合$\{\epsilon, 2, 3\}$,
$S \cdot \Sigma_{I}$可以看做集合
$\{1, 2\cdot 1, 2\cdot 2, 2\cdot 3, 3\cdot 1, 3\cdot 2, 3\cdot 3\}$,
$E$可以看做集合$\{1,2,3\}$, 
$T$可以看做一个从集合$S\cup S \cdot \Sigma_{I}$到集合$E$的映射,
例如,$T(2\cdot 1, 3) = 2$,其中,$2\cdot 1 \in S \cdot \Sigma_{I}$,
$3\in E$。
\end{example}

\begin{table}
\begin{center}
\caption{图\ref{example-mealy}中Mealy状态机对应的观测表}
\label{table-1}
\begin{tabular}{cc|c|c|c|c}
	\cline{3-5}
	& & \multicolumn{3}{ c| }{$E$} \\
	\cline{3-5}
	& & 1 & 2 & 3 &  \\
	\cline{1-5}
	\multicolumn{1}{ |c  }{\multirow{3}{*}{$S$} } &
	\multicolumn{1}{ |c| }{$\epsilon$} & 2 & 3 & 1 &   \\
	\cline{2-5}
	\multicolumn{1}{ |c  }{}                        &
	\multicolumn{1}{ |c| }{2} & 3 & 1 & 2 &   \\
	\cline{2-5}
	\multicolumn{1}{ |c  }{}                        &
	\multicolumn{1}{ |c| }{3} & 1 & 2 & 3 &    \\
	\cline{1-5}
	\multicolumn{1}{ |c  }{\multirow{7}{*}{$S\cdot \Sigma_{I}$} } &
	\multicolumn{1}{ |c| }{1} & 2 & 3 & 1 & \\
	\cline{2-5}
	\multicolumn{1}{ |c  }{}                        &
	\multicolumn{1}{ |c| }{$2\cdot 1$} & 3 & 1 & 2 &  \\
	\cline{2-5}
	\multicolumn{1}{ |c  }{}                        &
	\multicolumn{1}{ |c| }{$2\cdot 2$} & 1 & 2 & 3 &  \\
		\cline{2-5}
	\multicolumn{1}{ |c  }{}                        &
	\multicolumn{1}{ |c| }{$2\cdot 3$} & 2 & 3 & 1 &  \\
		\cline{2-5}
	\multicolumn{1}{ |c  }{}                        &
	\multicolumn{1}{ |c| }{$3\cdot 1$} & 1 & 2 & 3 &  \\
		\cline{2-5}
	\multicolumn{1}{ |c  }{}                        &
	\multicolumn{1}{ |c| }{$3\cdot 2$} & 2 & 3 & 1 &  \\
		\cline{2-5}
	\multicolumn{1}{ |c  }{}                        &
	\multicolumn{1}{ |c| }{$3\cdot 3$} & 3 & 1 & 2 &  \\
	\cline{1-5}
\end{tabular}
\end{center}
\end{table}



一个观测表是闭合的,当且仅当:
$\forall s\in S\cdot \Sigma_{I}. \exists t\in S. \forall r\in E.
T(s, r) = T(t, r)$。
直观上,一个观测表是闭合的,那么对所有元素$s \in S\cdot \Sigma_{I}$,
存在一个元素$t\in S$,使得在映射$T$下,所有输入有相同的输出。
表\ref{table-1}所示的表是一个闭合的观测表。

观测表的另外一个性质是$S$集合中不存在等价的元素,既任意两个元素在同一个映射下,都具有不同的输出。
因此,在给定的字典序关系下,$S$集合具有最小性。

\begin{definition}{(字典序)}

	字典序$<: \Sigma_{I}^{*} \times \Sigma_{I}^{*} \rightarrow Bool$是一个定义在字符序列上的二元关系。
	给定字符序列$s = a_0\cdots a_n$,$t = b_0\cdots b_n$, $s < t$当且仅当
	\begin{enumerate}
		\item $|s| < |t|$, 如果$|s| \neq |t|$;
		\item $a_{j} <_{a} b_{j}$
		如果$ a_{j} \neq b_{j} \wedge
		\forall 0 \leq i < j < |s|. a_{i} = b_{i}$;
		\item $false$, 其他情况。
	\end{enumerate}
	其中,$<_{a}$是任意一个在英文字母上的全序关系。
\end{definition}



\begin{algorithm}
\caption{$L^{*}$Mealy状态机学习算法: 观测表构造}
\label{algo-1}
\begin{algorithmic}
\STATE $S:=0$;
\STATE $S\cdot \Sigma_{I} := \{\epsilon\}$;
\STATE $E:=\Sigma_{I}$;
\WHILE{$\exists s \in S\cdot \Sigma_{I}.(\forall t\in S\cdot \Sigma_{I}.s \leq t) \wedge	(\forall t\in S.\exists r\in E. T(s,r)\neq T(t,r))$}
\STATE  将$s$加入$S$;
\FORALL{$a\in \Sigma_{I}$}
\STATE $S\cdot \Sigma_{I}$ = $S\cdot \Sigma_{I} \cup s \cdot a$;
\FORALL{$u\in E$}
\STATE 计算$s\cdot a\cdot u$的响应;
\STATE 获取响应的后缀$x$;
\STATE 更新$T(s\cdot a, u) = x$;
\ENDFOR
\ENDFOR
\ENDWHILE
\STATE 执行采样以及算法\ref{algo-2},如果存在反例。
\end{algorithmic}
\end{algorithm}


$L^{*}$状态机算法主要包含了三个步骤:

%\begin{enumerate}
%	\item

	(1)将集合$S$,$S\cdot \Sigma_{I}$初始化为只含有空字符串的集合,
	将集合$E$初始化为$\Sigma_{I}$,
	随后,算法构造闭合观测表。构造过程如算法\ref{algo-1}所示。
%	\item

	(2)算法依据第一步构造的闭合观测表,初步推断出一个状态机。构造方法如下:
	集合$S$中的元素表示状态机的状态,集合$S\cdot \Sigma_{I}$中的元素表示
	状态机的迁移关系,映射$T$的值域表示状态机的输出。
%	\item

	(3)算法将第二步构造的状态机输入给一个黑盒机器,称作“指导者”,
	该机器将判定上述构造的状态机是否正确的,
	并且能够在不正确的情况下,输出一则反例信息。
	反例信息是一个不满足上述状态机的协议实现输入输出消息序列。
	在具体实现中,步骤三中的黑盒机器通常通过采样方法生成一系列协议的输入输出消息序列
	算法随后将依据该反例信息,重新构造更加精确的状态机。
	构造过程如算法\ref{algo-2}所示。
%\end{enumerate}


\begin{algorithm}
	\caption{$L^{*}$Mealy状态机学习算法: 反例处理}
	\label{algo-2}
	\begin{algorithmic}
		\STATE $r, x$分别表示反例和相应的输出;
		\STATE $r = p\cdot s$, 其中,$p \in S \cup S\cdot \Sigma_{I}$,并且$p$是满足上述要求的最长的前缀;
		\FORALL{$s$的后缀$u$}
		\STATE $E = E \cup u $
		\FORALL{$t \in S \cup S\cdot \Sigma_{I}$}
		\STATE 计算针对$t\cdot u$的输出,并提取后缀$y$;
		\STATE $T(t,u) = y$;
		\ENDFOR
		\ENDFOR
		\STATE 执行算法\ref{algo-1}的内循环,如果$T$改变。
	\end{algorithmic}
\end{algorithm}

算法\ref{algo-2}描述了反例的处理。当一个反例产生时,算法找到了最长的前缀,并对其进行剪枝。余下的前缀加入观测表的列集合$E$中,表示所有对于新增前缀的行扩展进行计算。

成员查询请求的数量取决于被推断的状态机的大小。为了减少成员查询的数量,本节提出了一种新的启发式应答预测,该方法利用了推断模型中的冗余。任何错误的预测将被具有准确度和可信度的抽样查询检测出。

\subsection{测试序列生成算法研究}
%本文改进了Chow的W方法\cite{Chow1978}构造测试序列。对于LearnLib,可以指定等价检查的深度:给定状态机的假设,W方法的上限设置为找到的状态数加上指定的深度。该算法仅查找其长度最多为设定上限的反例跟踪,如果无法找到,则假定状态机的当前假设与实现的假设等效。如果实际状态机的状态数多于找到的状态数加上指定的深度,则此假设是正确的。 W方法非常强大,但在性能方面成本很高。因此,需要改进算法以利用学习的系统的特性。

本节基于Chow的$W$方法\cite{Chow1978}$W$方法研究了改进的测试序列生成算法。要对协议实现进行状态机学习,由于具体的状态机是未知的,在初始阶段并未构造出状态机模型,因此需要近似检查模型等价性,预估状态机模型,亦即根据模型反馈不断修正模型,这需要构造测试序列。如何生成合适的测试序列,是分析中需要解决的关键问题。基于有限状态机的测试序列生成方法主要有四种:转换回路方法(TransitionTour,简称T方法)、区分序列方法(Distinguishing Sequences,简称DS或D方法)、特征集合(Characterizing Set,简称W方法)和唯一输入输出序列(UniqueInput/Output Sequences,简称UIO或U方法)。首先介绍W方法。

\textbf{$W$方法简介}

$W$方法最早是由Chow\cite{Chow1978}
提出的,并被应用于电信系统的测试。该方法采用了自动机的理论,具有 3 个特征:(1)  仅仅检测系统设计中的控制结构;(2)  规格说明不必是可执行的;(3)  测试序列一定能揭示控制结构中的错误,并能解释错误的来源。
W方法并不需要使用状态位信息,而是采用分离序列(separating sequence)来区分
自动机中的每个状态。

\begin{definition}{(分离序列)}

对于$S$中的任意两个状态$s$和$t$,他们的分离序列是$x\in I^{*}$,以便使得$\lambda(s,x)=\lambda(t,x)$。形式描述为:
$\forall s,t:S,\exists x:I^{*}\bullet (\lambda(s,x)\neq \lambda(t,x))$。
\end{definition}

W 方法需要构造一个由分离序列组成的特征集W(characterizing set)和一个由输入序列组成的迁移覆盖集 P(transition cover set),再由这两个集合产生测试输入和预期输出。

\begin{definition}{(迁移覆盖集)}

Mealy状态机$M_s$的迁移覆盖集 P 是一个由输入序列构成的集合,以便对于$S$中的任意状态$s$和$I$中的每一个输入$a$,都存在一个输入序列$x\in P,x$ ,$x$从$M_s$的初始状态$s_0$开始到$\lambda(s,a)$产生的迁移结束。形式描述为:
$\forall s:S,\forall a:I\bullet (\exists x,y:P\bullet (x=y.a\cap \delta(s_0,y)=s)\cap \lambda(s,a))$
\end{definition}

\begin{definition}{(特征集)}

$M_s$的特征集$W$是一个由输入序列构成的集合,以便对$S$中的任意状态$s$和$t$,$W$中都存在一个输入序列$x$使得$\lambda(s,x)\neq \lambda(t,x)$。

\end{definition}

W方法通过广度优先遍历的方法,构造一棵测试生成树,并由树的根结点到每一个叶子的输入序列构成了$P$集;同时,对状态集合$ S $中的每一对状态$ s $和$ t$,寻找这两个状态的划分序列,并由这些划分序列构成$ W $集。W 方法得到的输入序列为
$r.P.W$,预期的输出序列为$\lambda(s,  r.P.W)$。$P$集的构造算法如下:

    \begin{algorithm}[htbp]
        \caption{构造迁移覆盖集$P$}
        \label{algo-pset}
        \begin{algorithmic} [1]%每行显示行号
        %\Require Input
        %\Ensure Output
 			\STATE 标记$s_1$为测试树$T$的根节点,并标记$T$的层次为1;
 			\IF {测试树T已被构造到第$k$层,下面开始;}
 			\STATE 构造第$k+1$层,从左到右,考虑$T$中第$k$层上的每个节点$t$;
 				\IF {如果$t$已经在$T$中的第$j$层上出现过,$j\leq k$}
 				\STATE $t$是终止节点,并作为$T$的一个叶节点;
 				\ELSE
 				\STATE 对每个输入$x$,如果状态机$M_s$从$t$到其他状态$s$,则将$s$作为$t$的一个字节点,并将
 				\ENDIF
 			\STATE 获得由测试树$T$的根节点到每个叶节点的变序列,并由这些边序列构造$P$集。
 			\ENDIF
        \end{algorithmic}
    \end{algorithm}

算法\ref{algo-pset}中包含一个个递归构造测试树的过程,导致空间复杂度的增加;另
外,该算法的时间复杂度为$O(n^2)$。此算法用于特征集$W$的构造。

    \begin{algorithm}[htbp]
        \caption{构造特征集$W$}
        \label{algo-wset}
        \begin{algorithmic} [1]%每行显示行号
        %\Require Input
        %\Ensure Output
 			\STATE 初始化$W=\oslash,B_1=S和r=1$;
 			\STATE 从$B_i$中获取两个状态$s$和$t$构造其分离序列$x$使得$\lambda(s,x)=\ne \lambda(t,x)$
 				\STATE 取$B_i$中两个状态$s$和$t$;
 				\STATE 以$s$为根节点,广度优先遍历$M_s$;
 			\STATE 增加输入序列$x$到$W$中,依据输出$\lambda(s_ik,x)$的不同,将$B_j$划分为更小的块$B_j1,B_j2,\ldots,B_jh$;
 			\STATE 统计$B_j$的个数为$r$;
 			\STATE 重复直至每个$B_i$都只有唯一的状态且$r=n$
        \end{algorithmic}
    \end{algorithm}
算法\ref{algo-wset}的时间复杂度取决于构造分离序列$x$的时间及循环时间。算法复杂度为$O(n^2)\sim O(n^3)$

由于W方法生成测试序列的复杂度约为$O(n^3)$,在实际情况中无法处理较多状态数的状态机。本节提出改进的测试序列生成算法,

\section{改进测试序列生成算法研究}
测试序列的复杂度主要体现在测试序列数量及长度上,考虑在生成状态覆盖集时首先改进W方法产生测试序列的长度。
本节对W方法进行了优化,针对W方法测试序列生成数量较大问题提出了套接字约简方法,该方法能够约简W方法中的所需构造的状态迁移序列,减少测试序列生成数量;针对测试序列长度较长问题,提出了检查点算法,该算法通过构造测试序列生成前缀树,约简生成的测试序列长度,提高状态机学习效率。

\subsection{套接字约简方法研究}
由于W方法在状态覆盖集构造过程中,
可以利用协议交互中的消息迹的实际意义对测试序列生成进行约简,通过特定的套接字序列,简化测试序列的生成。
在协议的实际运行中,存在部分协议状态对所有消息只给出一个响应,如连接断开状态。这种状态对后续测试无作用,但是W方法会将其作为受测系统响应,继续构造测试序列并进行学习。此时可以这部分状态集合定义为$S_u={s|\forall x\in P,\lambda(x,s)=}$可约简,因此首先分离状态覆盖集$S$及特征集$\mathsf{\mathcal{W}}$,
可以通过构造状态机的部分特征集,生成测试序列,该算法由两个阶段组成:

\textbf{第一阶段}:

计算假设模型$\mathsf{\mathcal{H}}$的迁移覆盖集$P$,状态覆盖集$S$,特征集$\mathsf{\mathcal{W}}$以及$M$的所有状态机识别.集${{W}_{i}}$。构造检查所有规范$M$定义的状态在实现中是否可识别,对每个$S$中的状态${{S}_{i}}$,它的识别集${{W}_{i}}$是确定的,并且$W$是一组至少包含所有${{W}_{i}}$输入序列的集合。设所有识别集${{W}_{i}}$的集合为$\mathcal{W}$ 。通过$P,S$以及$W_i$的选择,可以得到不同的测试序列。
第一阶段中的测试序列由$S,\mathsf{\mathcal{W}}$生成。每个规范的状态${{S}_{i}}$由$\mathsf{\mathcal{W}}$集检测。如果所有的测试都成功了,则$M{{\approx }_{S.W}}\mathsf{\mathcal{H}}$ ,此时协议实现中的状态数与规范$M$中的状态数相等。
$\mathsf{\mathcal{H}}$检测对每个状态${{s}_{j}}$是否可以由最小识别集${{W}_{j}}$识别。同时,对从初始状态到这些状态$\mathsf{\mathcal{H}}{{\mathsf{}}_{j}}$的迁移进行检查。

\textbf{第二阶段}:

对所有协议规范所确定的,但在第一阶段中未检测的迁移进行检测。
此时构造测试序列属于$P$而不属于$S$,记$R=P-S$,则此时有
\[R\otimes \mathsf{\mathcal{W}}=\bigcup\limits_{p\in \mathsf{\mathcal{R}}}{\{p\}.}{{W}_{i}}\]
这里${{W}_{j}}$是$\mathsf{\mathcal{W}}$中${{S}_{j}}$的识别集。这个阶段对剩余的迁移关系进行检测。进行完两个阶段即可生成正确的受测系统状态机。


%实际分析的目标系统称为被测系统(system under test,SUT),在状态机推断中,并不知道协议实现库实际的状态机,此时需要构造测试序列作为等价查询对SUT进行测试以得到反例。


%通过Wp方法产生的测试集的复杂度为$O({{n}^{2}}|\Sigma {{|}^{(m-n+1)}})$,其中 是状态机的输入字母表, 是目标系统的状态数上界, 是状态机的状态数。因此,当Wp方法的状态数 较大时,会造成测试序列数量呈指数性增长,不利于大状态数系统的学习。


\subsection{检查点算法研究}
考虑到在构造测试序列时,受测系统有时需要处理大量具有相同前缀的查询,从初始状态开始处理查询的话,在对查询的相同前缀进行处理时,受测系统进行相同的状态迁移。如果每次都对受测系统进行重置,需要大量的重复测试过程,因此需要对测试序列的处理进行优化,在处理受测系统时,对其可能遇到的前缀时的状态及测试序列前缀进行记录,称这个记录点为检查点。可以通过构造前缀树(trie)用来存储之前执行过的查询,实现检查点算法。
树中每个节点记录一个之前的输入,相应的输出以及当时的状态的相关信息。在树的输出查询中,从根节点开始对每个查询中的输入步骤选择相应的子节点。在这个过程中能够找到对每个输入步骤的相应的输出。一个查询由已知输出的前缀,以及其后缀还未执行过未知其输出的后缀组成。算法描述如表\ref{checkpoint}所示:


\begin{algorithm}[htbp]
	\caption{检查点算法}
    \label{checkpoint}
    \begin{algorithmic}[1]%每行显示行号
        \IF  {查询已经在完全在前缀树中}
        \STATE 可在前缀树中找到对应输出
        \ELSE
        \STATE 寻找查询的最长已执行前缀的检查点c
        	\IF {没有已知检查点 c}
        	\STATE 使用初始检查点
        	\ENDIF
        	\STATE 重启检查点c来保存受测系统的状态
        	\STATE 执行查询的剩余部分
        		\IF {需要记录查询}
        		\STATE 报出受测系统的新状态
        		\STATE 在前缀树中加入新的检查点
        		\ENDIF
        \RETURN
        \STATE 前缀树中检查点c中的输出+新观测得到的输出
        \ENDIF
        %\IF {没有已知检查点}
        %\STATE {使用初始检查点}
        %\ENDIF
                	%\State 重启检查点c来保存SUT的状态
                	%\State 执行查询的剩余部分
                	%\If {需要记录查询}
                	%\State 保存SUT的新状态
                	%\State 在前缀树中加入新的检查点
                	%\EndIf
                	%\Return 前缀树中检查点c中的输出+新观测得到的输出
                %\EndIf\State
    \end{algorithmic}
\end{algorithm}


检查点算法能够通过构造前缀树,存储已知前缀,在生成测试序列的数量不变的前提下,将相同前缀的测试序列带来的系统状态迁移次数减少,从而使得状态机生成时间减少。



\section{协议实现的状态机学习设置}

协议实现的状态机学习技术包括三个部分:指导者,映射器和受测系统。其中指导者中包含,改进的测试序列生成算法与$L^{*}$算法。指导者生成抽象输入,
代表协议消息。映射器把这些消息转换为格式良好的协议数据包并将它们发送到受测系统。 受测系统将响应数据包发送回映射器,并反过来
将这些数据包转换为抽象输出。 映射器再给学习者发回抽象输出。
学习者使用一个Java库LearnLib来实现基于L *的算法,用于学习Mealy状态机。受测系统可以是任何协议实现。


%%%安全协议实现的状态机学习技术由三个部分组成:指导者、受测协议实现以及生成的状态机模型。其中指导者中包含,改进的测试序列生成算法与$L^{*}$算法。受测协议实现分为判定器及映射器。



整体架构设计如图\ref{fig-learningset}所示。

\begin{figure}[htp]
	\centering
	\includegraphics[width=15cm]{data/content/chap04/learningconfig}
	\caption{协议实现状态机学习框架设置}
	\label{fig-learningset}
\end{figure}

%%%具体流程为:首先通过建立预估的近似状态机模型,模型通过改进的测试序列生成算法生成等价查询,等价查询通过映射器用于在状态机模型进行等价查询并且不一致时给出反例,$L^{*}$算法接受反例,并生成新的模型。根据新的模型,向映射器发送成员查询。映射器接受该成员查询后,将其转化为协议实现可接收的交互报文,并发送给受测系统。受测系统给出响应,映射器将其抽象为指导者的输入字母,并基于此反馈输出字母用于修正模型生成。
具体流程为:首先通过建立预估的近似状态机模型,模型通过套接字约简方法构造生成测试序列。生成等价查询,等价查询通过映射器用于在状态机模型进行等价查询并且不一致时给出反例,$L^{*}$算法接受反例,并生成新的模型。根据新的模型,向映射器发送成员查询。映射器接受该成员查询后,将其转化为协议实现可接收的交互报文,并发送给受测系统。受测系统给出响应,映射器将其抽象为指导者的输入字母,并基于此反馈输出字母用于修正模型生成。


本节状态机学习设置是为了指定学习过程中用到的参数,这些参数将决定学习到的模型是否能够完整表征协议实现。主要设置学习算法、检查深度、输入输出字母表等内容。

\textbf{学习算法}

为了推断TLS协议实现的米利状态机,使用了开源的状态机学习框架LearnLib\cite{Raffelt2009},它使用了L*算法进行学习。受测系统对于测试者是黑盒,由于无法得知协议库的实际执行状态。因此必须通过映射器向LearnLib提供可以发送到受测系统的消息列表以及将受测系统重置为其初始状态的命令。
通过发送消息和重置命令的序列,利用测试工具将抽象消息从输入字母表转换为可以发送到被测系统的具体消息,LearnLib尝试根据从受测系统收到的响应为状态机提出假设。然后检查这些假设是否与实际状态机等效。如果模型不相等,则返回一个反例,LearnLib将使用它来重新修改其假设模型。

\textbf{检查深度}

测试中首先需要指定等价检查的深度:给定状态机的假设 ,需要将测试上限设置为找到的状态数加上指定的深度。该算法仅查找其长度最多为设定上限的反例跟踪,如果无法找到,则假定状态机的当前假设与实现的假设等效。如果实际状态机的状态数多于找到的状态数加上指定的深度,则认为此假设是正确的。在实际中,检查深度的增长会明显延长测试时间,因此测试深度不宜过高。

\textbf{输入输出字母表}

为了使用LearnLib,还需要设定一个可利用的输入字母表,它可以将实际发送和收到的消息抽象为一个字符串作为LearnLib的输入,根据TLS握手阶段交互的消息,设定客户端、服务器端的输入输出字母表为:客户端握手,客户端证书,完成,客户端密钥交换,客户端证书验证,更改密码规范,应用数据;服务器端握手,服务器端证书,完成,服务器端密钥交换,服务器端证书验证,更改密码规范,应用数据。由于有警告协议中的消息格式只在响应中,因此在输出表中加入:空报文,解密失败以及连接中断。如图\ref{fig:chap04-alphabet}是服务器端可收到的字母表。

\begin{figure}[htp]
	\centering
	\caption{输入输出字母表}
	\lstinputlisting[label={list:design,Language = C}]{data/content/chap04/alpha.c}
	\label{fig:chap04-alphabet}
\end{figure}


\textbf{映射器}

映射器作用是将具体协议报文抽象为输入输出字母,并将LearnLib的成员查询转化为具体的报文发送给协议实现,通过Python实现。
%test harness%

\textbf{可约简套接字}

(1)连接中断
在寻找反例时,连接关闭后的追踪是没有意义的,在与受测系统的交互中,一旦会话结束,但是受测系统仍然会产生连接断开的响应,Wp方法会将此响应作为目标系统的输出构造等价查询寻找反例,而这些是无效的测试序列,需要终止后续测试序列生成。

(2)密钥重协商
在协议的会话中,可以通过发送密钥重协商消息进行密钥更改,这个过程可以看作是将协议状态重新迁移到了握手协议中的更改密码规范状态,二者在其后的测试序列生成一致,是多余的测试序列,可以停止生成重协商之后的测试序列生成。

(3)异常消息警告
异常消息警告包括协议连接断开通知、关键警告以及警告。其中连接断开通知和关键警报会造成连接断开,与情况(1)中的连接中断类似,终止后续测试序列生成。
该改进算法利用了协议交互中的连接特性,利用特定套接字对测试序列生成进行了约简。



\section{实验与结果分析}
本节利用提出的状态机学习算法对三种TLS协议实现GnuTLS、OpenSSL以及NSS进行状态机模型提取。

\subsection{实验设计}

\textbf{实验名称}:对TLS协议实现GnuTLS、OpenSSL以及NSS进行状态机模型提取。

\textbf{实验对象}:本实验分析了三种不同的TLS实现,如表\ref{table:chap04-implementations}所示。
\begin{table}[htb]
  \centering
  \caption{测试的TLS实现}
\begin{tabular}{|c|c|c|}
  \hline
  % after \\: \hline or \cline{col1-col2} \cline{col3-col4} ...
  名称 		& 版本 		& URL  \\
  \hline
  GnuTLS 	& 3.3.6 		& http://www.gnutls.org/  \\
  \hline
  OpenSSL 	& 1.0.1g 	& https://www.openssl.org/ \\
  \hline
  NSS 		& 3.16.3    & https://developer.mozilla.org/en-US/docs/
Mozilla/Projects/NSS \\
  \hline



\end{tabular}
\label{table:chap04-implementations}
\end{table}


GnuTLS是一个广泛应用于SASL服务器的开源TLS实现。
Openssl是应用最为广泛的开源TLS实现,通常用于安全通信中,支持Apache等服务器提供https服务。
NSS是由Mozilla开发的TLS实现,它广泛应用于主流的web浏览器中,如FireFox,Chrome还有Opera。通常用作客户端。

\textbf{实验环境}:4核Intel Core™ i7-4710MQ CPU,主频2.5GHz,8GB内存,搭载Ubuntu 14.04 64位操作系统。


使用检查深度为3来对协议实现进行测试。下面依次分析通过状态机学习算法学习到的三种TLS实现的状态机。

\textbf{实验步骤}:
在分析得到的模型时,首先人工查看是否有比预期更多的路径导致应用程序数据的成功交换。其次确定模型是否包含多于必要的状态,并识别意外或多余的过渡状态。还需要检查可以指示异常状态的消息。如果遇到任何异常,那么就要更深入的分析以确定原因及影响。可以观察到服务器端实现的状态机模型都有所不同,这意味着TLS实现在实现过程由于开发者对协议规范的理解不同,那么在实际应用中,就存在着潜在的安全威胁。


\subsection{结果分析}
本节对实验结果进行描述并对改进的状态机学习算法进行分析。
\subsubsection{实验结果}
通过改进的状态机算法成功地提取了GnuTLS、OpenSSL及NSS这三种TLS协议实现的状态机模型。
\textbf{GnuTLS}
图\ref{fig-gnuserver}显示了通过状态机学习算法学习得到的GnuTLS 3.3.6的服务器状态机模型。
\begin{figure}[htp]
	\centering
	\includegraphics[width=16cm]{data/content/chap04/gnuserver.pdf}
	\caption{GnuTLS3.3.6服务器状态机模型}
	\label{fig-gnuserver}
\end{figure}

图\ref{fig-gnuclient}显示了通过状态机学习算法学习得到的GnuTLS 3.3.6的客户端状态机模型。
\begin{figure}[htp]
	\centering
	\includegraphics[width=16cm]{data/content/chap04/gnuserver.pdf}
	\caption{GnuTLS3.3.6客户端状态机模型}
	\label{fig-gnuclient}
\end{figure}
在此模型中,有两条路径可以成功交换应用程序数据:常规的没有客户端身份验证的路径和一个在握手期间发送的空客户端证书。由于不需要客户端身份验证,因此两者都是可接受的路什么是明确的是,有比预期更多的州。握手协议然后继续,但最终导致致命警报(“内部错误”)以响应完成消息(来自状态8)。从握手协议中的每个状态,可以通过发送HeartbeatRequest消息转到“阴影”路径中的相应状态。此行为是由安全错误引入的。另外,存在冗余状态5,其在发送ClientHello消息时从状态3和9到达。从状态5开始,对发送的所有后续消息给出致命警报。人们可能会收到一条错误消息,以响应ClientHello消息本身。可以利用此错误来有效地绕过依赖于比较握手中消息的键控哈希的完整性检查:同时在客户端重置此缓冲区的同时能够成功完成握手协议,但随后在先前交换的握手消息上没有提供完整性保证。

\textbf{OpenSSL}

图\ref{fig-opensslserver}显示了通过状态机学习算法学习得到的OpenSSL1.0.1g的服务器状态机模型。
\begin{figure}[htp]
	\centering
	\includegraphics[width=15cm]{data/content/chap04/opensslserver}
	\caption{OpenSSL1.0.1g服务器模型}
	\label{fig-opensslserver}
\end{figure}


图\ref{fig-opensslclient}显示了通过状态机学习算法学习得到的OpenSSL1.0.1g的客户端状态机模型。
\begin{figure}[htp]
	\centering
	\includegraphics[width=15cm]{data/content/chap04/opensslclient}
	\caption{OpenSSL1.0.1g客户端模型}
	\label{fig-opensslclient}
\end{figure}


获得的模型包含比预期更多的状态,但仅包含一条成功交换应用程序数据的路径。
该模型显示协议可以通过发送两个ClientHello消息开始,但是该消息数量不能超过两个。 在第二个ClientHello消息之后,模型中则没有成功交换应用程序数据的路径。 这是因为每次发送ClientHello时OpenSSL都会重置包含握手消息的缓冲区,因此不会接受在握手结束时计算的哈希值,并且会通过警报响应状态9中的Finished消息。


\textbf{NSS}

图\ref{fig-nssserver}显示了通过状态机学习算法学习得到的NSS的服务器端状态机模型。
\begin{figure}[htp]
	\centering
	\includegraphics[width=16cm]{data/content/chap04/nssserver}
	\caption{NSS服务器模型}
	\label{fig-nssserver}
\end{figure}


图\ref{fig-nssclient}显示了通过状态机学习算法学习得到的NSS的客户端状态机模型。
\begin{figure}[htp]
	\centering
	\includegraphics[width=15cm]{data/content/chap04/nssclient}
	\caption{NSS客户端模型}
	\label{fig-nssclient}
\end{figure}

通过对NSS状态机模型分析,模型中只有一条能够导致数据成功交换的路径。总的来说,NSS对所有不符合协议规范的数据都给出了警告。但是有个例外是,当发送客户端握手前,如果发送非握手消息至服务器,那么服务器就会打开连接,但不继续响应。



特别是,协议状态机学习涉及一种被动测试:在状态机学习期间执行的测试包括错误测试,即消息以意外命令发送的测试,人们期望这会导致连接中断,而如果连接并没有按照规范设定的中断的话,则需要深入调查其原因。图\ref{fig-opensslserver}是OpenSSL 1.0.1g服务器端学习到的状态机模型。对此状态机进行分析,可以看到一条可疑路径(状态0,1,6,9,12)。
如
\subsubsection{改进的测试序列生成算法效果分析}
使用检查深度为3来对协议实现进行测试。
表\ref{table:chap04-testresult}为使用指定检查深度为3的改进测试序列生成方法对输入的常规字母表的协议服务器端进行自动测试的结果。

\begin{table}[htb]
  \centering
  \caption{检查深度为3的改进Wp方法对协议服务器端进行自动测试结果}
\begin{tabular}{|c|c|c|c|c|c|}
  \hline
  % after \\: \hline or \cline{col1-col2} \cline{col3-col4} ...
  受测服务器端	& 算法 		& 状态数  	&  时间/min 	 	&成员查询 	&等价查询    \\
  \hline
  GnuTLS 3.3.6	&改进算法	&  12		&	45		&	1545		&	4862		    \\
  				&套接字约简	&	12		&	53		&	1545		&	4862 	\\
   				&检查点算法  &	12		&	72		&	1545	&	6320		\\
   				&原始算法    &	12		&	92		&	1545	&	6320		\\
  \hline
  OpenSSL 1.0.1g 	& 改进算法 	& 16    &	37		&	2508		&  	7592		\\
    				&套接字约简	&16			&42			&	2508		&	7592		\\
    				&检查点算法	&	16		&	69		&	2508		&	10587	\\
    				&原始算法	&	16		&	83		&	2508		&	10587		\\
  \hline
  NSS 3.16.3 		&改进算法    & 	8		&	43		&	592		&	9736		\\
    				&套接字约简	&	8		&	48		&	592		&	9736		\\

  					&检查点算法  &	8		&	87		&	592		&	14561		\\
   					&原始算法	&	8		&	106	&	592		&	14561		\\

  \hline
\end{tabular}
\label{table:chap04-testresult}
\end{table}

下面依次对各种方法进行分析。
%\subsubsection{改进的测试序列生成算法效果分析}

\textbf{套接字约简方法分析}

可以看出通过套接字约简方法,生成的等价查询数显著下降,这意味着改进的测试序列生成方法能够有效约简学习状态机过程中所需的测试序列数量。从学习状态机模型所用时间上看,生成模型所用时间显著下降,这意味着该方法通过TLS的特定套接字,极大地减少了测试过程中的冗余信息。


\textbf{检查点算法分析}

实验结果表明,虽然只应用检查点算法的改进算法无法减少所需的成员查询和等价查询数量,但是可以明显减少状态机学习所用时间。这说明通过检查点算法,能够显著降低受测系统在测试过程中对测试序列的响应时间,加速反例生成,提升状态机学习效率。分析其原因,该算法在生成测试序列的数量不变的前提下,将相同前缀的测试序列带来的系统状态迁移次数减少,从而使得状态机生成时间减少。

\textbf{结合方法分析}

二者结合的改进算法所需时间最少,效率最高,这是由于两种方法分别从不同的方面对算法效率进行了提高。结合方法能够利用套接字约简方法对状态机学习过程中所需的测试序列数量减少,同时检查点算法能够提高反例生成时间,整体对状态机学习效率提升显著。实验证明,和原始算法比,改进算能够显著减少生成状态机所用等价查询数量,如图\ref{fig-member}。

\begin{figure}[htp]
	\centering
	\includegraphics[width=10cm]{data/content/chap04/member}
	\caption{改进算法与原始算法等价查询数量对比}
	\label{fig-member}
\end{figure}

改进算法能够显著降低生成协议实现状态机所需时间,如图\ref{fig-time}。

\begin{figure}[htp]
	\centering
	\includegraphics[width=10cm]{data/content/chap04/time}
	\caption{改进算法与原始算法生成状态机所用时间对比}
	\label{fig-time}
\end{figure}







\section{本章小结}


本章提出了一种改进的测试序列构造方法,它利用套接字约简,能够有效减少在对目标系统黑盒测试中所用到的测试序列数量,减少运算量;以及检查点算法,降低分析的运行时间,从而提高状态机学习效率,在未知目标系统规范的黑盒测试中具有很好的效果,从而使得分析复杂状态机运行成为可能。实验结果表明,完整改进算法效率最高,即结合了套接字约简的等价查询数减少,又通过检查点算法提高了测试效率。整体上看,套接字约简对学习效率的提升效果好于检查点算法。
本章结合状态机学习的L*算法以及等价查询算法Wp方法,成功学习到了GnuTLS,OpenSSL,NSS三种TLS协议实现的状态机模型。并以此为基础对TLS协议实现进行深入的分析。通过状态机学习技术来测试协议实现,能够有效发现其中存在的逻辑缺陷。
