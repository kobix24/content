%!TEX root = C:\Users\kobix\Documents\gradu\thesis1\thesiscc.tex
\chapter{绪 论}

\textbf{协议},是网络协议的简称,是指双方实体完成通信或服务所必须遵循的规则和约定。协议定义了数据单元使用的格式,信息单元应该包含的信息与含义、连接方式、信息发送和接收的时序,从而确保网络中数据顺利地传送到确定的地方。用于确保网络安全可靠的协议称为\textbf{安全协议}。安全协议是建立在密码体制基础上的一种交互通信协议,它运用密码算法和协议逻辑来实现认证和密钥分配等目标。安全协议是网络安全的一个重要组成部分。
协议的安全性分析是验证其是否达到其声明的安全目标。通常的协议安全性分析的步骤是确定协议的安全目标、确定协议安全性的衡量标准、设定协议分析的出发点或基本前提。

本章介绍论文的研究背景与意义,协议形式化分析的研究现状以及本文的主要工作及内容安排。
\section{论文研究背景与意义}
协议是网络的基本组成部分,其安全性对网络安全具有重大影响。随着云计算、物联网、车载网、互联网+、智慧城市等更广泛的应用,网络信息技术极大地推动了社会的进步,但也带来了潜在的安全风险,必须重视协议在网络安全中的作用。许多关键服务如远程医疗、网上银行、电子商务、机器自动协同、移动计算和云计算都离不开安全协议,最典型的就是能够在数字实体之间建立加密信道的TLS。 随着信息技术的发展,安全协议的应用将会更加广阔。除保证连接完整性和机密性外,为了保证网络通信的安全性,大量的安全协议被提出、设计和部署于系统中,以确保信息传输的机密性、远程登录的认证性、电子商务的不可否认性等。比较著名的安全协议有 Kerberos、SSL/TLS、WPA/WPA2、SSH 和 IPSec 等,它们被广泛的用于各类工业控制系统、无线通信网络和互联网中。研究表明,超过60%的Internet连接是通过基于TLS的安全HTTPS协议实现的\cite{10.1145/2674005.2674991,8406957}。


从协议的设计到实现再到应用的过程上看,协议的内涵包括规范、实现以及配置三个层次,如图\ref{fig-layer}所示。协议规范层重点考虑协议实体间的消息时序关系,实体内部运算细节,例如具体的消息长度、采用的加密算法、加密模式、分组填充方式、协议版本兼容、子协议交互等等。表现为使用函数符号抽象实体的内部计算。规范层的协议通常为消息交互图(Message  Switch Chart, MSC) ,其漏洞主要为逻辑漏洞和部分密码算法漏洞,目前的符号方法和计算方法均针对规范层协议。
协议实现层主要考虑协议采用具体编程语言的软件实现,包括协议发送与应答消息的具体实现,协议客户端与服务器端对消息的处理。其安全漏洞除了包含规范层漏洞外,还包括实现协议功能的软件的漏洞,如缓冲区溢出、内存泄露等。例如“心脏出血”漏洞与传输层安全协议(Transport Layer Security,TLS)规范无关,只存在某些版本的TLS协议实现OpenSSL中。协议的安全性形式化分析主要集中在分析协议规范的高级抽象模型上目前很少有针对实现层的形式化分析方法。通常对其的分析只能聚焦于传统软件安全性分析层面,很难找到其中存在的逻辑性问题。
协议配置层主要通过网络参数配置,使得网络中各节点能够彼此正确地通信对方,主要包括网络IP地址、MAC消息、证书的配置等。其安全漏洞主要对应于错误的参数配置,不涉及协议自身的安全性。
\begin{figure}[htp]
\centering\includegraphics[width=7cm]{data/content/chap01/protocollayer1}
\caption{协议分析层次}
\label{fig-layer}
\end{figure}

形式化分析方法采用数学模型对安全协议进行建模,具有可靠的理论基
础,再辅以计算机自动推理验证技术,可以自动或辅助证明协议的安
全性或发现其存在的安全隐患。例如G.Lowe基于CSP建模方法和FDR
模型检测工具,发现了 Needham-Schroeder 公钥协议中存在 17 年的
漏洞。由于基于严格的数学逻辑模型和推理方法,形式化分析的结论
被业界广泛认可。是否经过形式化分析,已成为一个安全协议安全与
否的重要指标。

研究者选择目标协议后对其进行形式化分析的常用处理流程如图\ref{fig-formalanalyse}所示。首先通过形式化的语言建立的形式化的协议模型,同时对协议声明的安全属性进行建模,然后通过证明安全或不存在敌手攻击的方式验证协议安全属性可满足。

\begin{figure}[htp]
\centering\includegraphics[width=15cm]{data/content/chap01/formalana}
\caption{协议形式化分析的流程}
\label{fig-formalanalyse}
\end{figure}


但是,协议安全性的形式化分析仍面临很多问题。首先,在基于符号的形式化方法中,协议的安全性分析都只考虑了最简单的敌手模型,这与现实中敌手具有层出不穷的攻击手段不符合。如何对敌手的真实能力进行描述, 一直是安全协议形式化研究工作的
重点。敌手攻击能力模型是协议形式化分析方法是否可靠的关键:敌手模型若比实际强大,将产生误报,安全的协议会被认为存在漏洞,导致可用的协议过于复杂;敌手模型若比实际弱小,则产生漏报,包含漏洞的协议会被认为安全,协议被部署到系统中会存在安全隐患。其次,目前协议形式化方法只能对特定类型的协议或者安全属性进行验证,没有适合所有协议类型的安全性分析方法。
最后,对协议实现的安全性形式化分析验证亟待研究。目前对协议的分析大多数分析是针对协议规范进行的,都是一次性分析了该规范的单一版本。但是基于规范的协议实现是不断动态变化的,必须验证这些更改不会对安全功能产生负面影响。因此,为了使形式化分析真正有用,需要将其更紧密地集成到协议的标准实现中。在实际使用中,协议实现架构复杂,与报文的消息交互以及具体的数据的控制流相关。基于具体执行的分析技术,由于计算量随着被测系统规模呈指数级增长,只能处理具有较少状态空间的协议实现,对于大规模状态空间的协议,很难穷尽其中的状态空间,建立完备的系统模型。

通过基于形式化方法的协议安全性研究,能够:(1)为协议的安全性分析提供了新的解决思路。协议的形式化分析兼具理论性及实用性,能够通过最有效的自动化分析方法,并尝试结合计算方法中的证明可靠性优势,拓展传统敌手模型,提高安全协议形式化符号分析技术中的分析准确性。其次,在安全协议的验证过程中,对于安全协议实现的逻辑漏洞的研究不足,考虑通过建立安全协议实现的状态机模型,对其进行形式化分析以获得全面可靠结论。
(2)为影响网络安全的协议安全漏洞提供现实的解决途径。随着技术的不断进步,智能终端能够通过多种网络接入技术,例如IEEE802.11、蓝牙、GSMGPRS、HTTPS等通信协议,实现随时交互的信息的强大能力,随之而来的是无处不在的信息泄露或被篡改的风险。 通过对基于形式化方法的协议安全性分析技术的研究,能够对协议形成系统全面的分析,及时找到潜在的安全风险,保障网络与信息的安全。









%%%本文以安全协议的形式化分析技术研究为背景,综合利用数学、计算机科学和现代密码学等学科的相关理论与方法,系统分析目前最新的安全协议形式化分析方法,在此基础上研究适用安全协议的分析模型,设计分析算法,能够对安全协议进行从规范层到实现层的系统分析。


\section{协议形式化分析研究现状介绍}
Roger Needham和Michael Schroeder最早将形式化分析的思想引入协议分析中,他们于1978年利用形式化思想,提出了著名的Needham-Schroeder安全协议用于通讯双方的身份认证,并于1989年被BAN逻辑\cite{Burrows1990}证明是正确的。1995年,Gavin Lowe采用了允许协议节点与敌手共谋的敌手模型,,发现了对该协议的攻击\cite{lowe1996breaking}。这推动协议的形式化分析进入了新的阶段。  如今在安全协议形式化分析研究领域,主要有符号化方法和计算方法两大分支。

\subsection{符号方法}
符号方法是利用符号化的公
式表征安全协议,并通过逻辑推导等方式来分析协议安全性的方法。 
符号化方法的基础是Dolev-Yao模型\cite{Dolev1982}。Dolev-Yao模型是由Dolev和Yao在1983年为验证安全协议的安全性而建立的,该模型对网络的攻击者进行了定义以及行为能力描述。Dolev-Yao 模型是符号化方法中最流行的敌手模型。人们几乎是同时开始了一般安全协议分析工具的开发,以便在Dolev-Yao模型的基础上对安全协议进行自动分析。最早用于安全分析的是由Millen开发的工具Interrogator\cite{1702206},它的原理是企图通过遍历整个状态空间,找到协议安全的漏洞,它的基本思想被后来的许多工具所继承。



\subsubsection{模态逻辑技术}
模态逻辑技术通过分析通信过程中双方发送和接收的消息, 根据一
系列的推理公理来判断安全协议是否能达到预期的目的。这种技术是一个演绎推理的过程。 在这种理论体系中, 推理过程大致可以分为以下
4 步进行。
(1) 将安全协议过程理想化, 进行形式化描述;
(2) 设定安全协议的初始化假设;
(3) 对安全协议的目标进行形式化描述;
(4) 从假设和协议过程的事实出发, 使用形式化逻辑对其进行推导分析, 得到最终状态时各个主体的
信仰和知识, 判断其是否达到安全目标。

 1989 年,Burrows、Abadi和Needham 提出了著名的BAN逻辑\cite{Burrows1990}。 BAN逻辑通过对协议过程和假设进行形式化,并通过一系列的推理准则进行推理来判断是否能得到预期结果。 BAN逻辑有力地推动了安全协议形式化分析的研究,并逐步演化出基于逻辑的方法\cite{gong1990reasoning,Abadi1991,Kunnemann},基于进程演算的方法\cite{Jeffrey2006,Abadi1999,gordon2001authenticity,gordon2002types},以及基于定理证明的方法形式化方法\cite{Paulson1997a,Paulson1997,Cortier2017}。

BAN 逻辑是最早提出的基于模态逻辑技术的形式化分析方法, 是安全协议分析的里程碑,因其简
单、直观、便于掌握使用的特点得到了广泛的应用。 BAN 逻辑是基于信仰的形式化分析方法, 具有 7 类 19
条推理规则, 通过协议过程中消息的接收和发送的事实, 来判断协议主体的最终信仰是否与预期结
果相符。在应用过程中, 人们也发现了 BAN 逻辑中存在的缺陷, 包括协议理想化过程不规范,协议的初始化假
设不合理,缺乏完备性,缺乏一个定义良好明确的语义等问题。BAN逻辑具有简洁直观的特点,并且容易理解也易于使用。

但是BAN逻辑也存在着一些缺点。BAN逻辑的敌手模型固定,在分析中不考虑被攻陷的网络节点。 Needham-Schroeder协议在BAN逻辑中被证明​​是正确的,因为该方法无法对中间人攻击进行建模。 同时BAN逻辑使用了相当弱的身份验证概念。 在BAN逻辑中,谓词的形式为“P believes X”。 此类谓词是使用诸如“如果$P$相信$P$和$Q$共享密钥$K$,并且如果P看到消息${X}_K$,那么$P$相信$Q$曾经说过$X$”等推导规则从一组假设中得出的。这种身份认证的条件无法满足认证性的要求。由于上述限制,研究者又开始研究其扩展和以解决这些问题, 基于 BAN 逻辑的类BAN 逻辑也相继提出。

GNY 逻辑\cite{mathuria1995automation}是第一个对 BAN 逻辑进行扩展的类 BAN 逻辑, 其各种规则总共合计 44 条。 GNY 逻辑试
图消除对主体诚实的假设、消息源假设和可识别假设, 具体从下面六个方面对 BAN 逻辑进行改进: 推广了
逻辑分析应用范围、增强逻辑分析能力、对形式化协议时的明文进行保留、增加了“拥有”集合、引入“可
识别”的概念以及引入“不是由此首发”的概念。
GNY逻辑试图针对 BAN逻辑的不足进行修正, 但是效果并不明显。 同时, 由于其本身过于复杂, 许多
学者认为 GNY 逻辑在实际应用中行不通。 但是 GNY 逻辑为 BAN 逻辑的改进指明了方向。

AT 逻辑\cite{Abadi1991}在 BAN 逻辑提出不久后出现。 AT 逻辑首次提出了逻辑系统的语义模型, 使表达能力更强。
AT 逻辑从语义角度出发, 改进了 BAN 逻辑, 同时给出了形式化语义, 并证明了其推理的合理性。 AT 逻辑
对 BAN 逻辑的改进具体包括四个方面: 从语义角度分析修改了 BAN 逻辑, 整理了 BAN 逻辑的推理规则,
除去了语义和实现细节混合的部分、对一部分逻辑构件使用了更加直接的定义、引进了全部命题连接词,
将推理规则改写成公式, 简化了推理规则以及给出了形式化语义, 并证明了推理系统的合理性。
相比 BAN 逻辑, AT 逻辑更接近传统的模态逻辑, 包含详细的计算模型和模型语义, 极大地推动了
BAN 逻辑的发展。 但同时, AT 逻辑也有不足之处, 比如未能提供基于公钥体系的分析方案, 以及个别公理
存在一定缺陷等问题。
SVO 逻辑\cite{Syverson1996}是 Syverson 和 Van Oorschot 综合了 BAN 逻辑、 GNY 逻辑、 AT 逻辑和 VO 逻辑的优点提
出的, 它为逻辑系统建立了合理的理论模型, 并在形式化语义方面对 AT 逻辑进行了重新定义。 其主要特
点体现在四个方面: 提出了较为清晰的模态理论语义、消除了逻辑形式化表述和推理规则方面可能引发的
模糊问题、具有极好的扩展能力以及简单实用的特点。
SVO 逻辑的提出, 标志着类 BAN 逻辑的成熟。

这些基于模态逻辑的方法有着自己的不足。主要表现为逻辑的抽象性较高,这种抽象性往往会掩盖(或丢掉)协议执行的状态信息,因而难以完全反映协议运行的全貌,有着自己的局限性。


\subsubsection{模型检测技术}

BAN逻辑之后,许多形式化方法逐渐被应用于对安全协议的分析上。从上个世纪90年代起,安全协议的形式化分析研究出现了空前的繁荣景象,其中一种有代表性的方法就是模型检测方法。模型检测(model checking)是一种强大的有限状态并发系统自动化验证技术,最早用于分析和模拟硬件工作过程。模型检测的基本思想是:检验一个结构是否满足一个公式要比证明公式在所有结构下均被满足容易得多,
进而面向并发系统创立了在有穷状态模型上检验公式可满足性的验证新形式。1996年,Gavin Lowe首次将模型检测器FDR用于安全协议验证,结合通信序列进程(CSP)安全协议模型成功发现了针对NSPK协议的攻击。这个结果对于应用模型检测的方法分析安全协议有着典型意义。


模型检测技术的基本思路是根据有限状态机理论, 通过定义状态集和状态迁移函数为安全协议系统
建立模型, 通过穷尽所有空间状态来判断某些特殊状态是否可达, 或者是否可以生成一条特殊的状态转移
路径, 来判断是否可以达到安全期望。 模型检测技术一般将安全协议看作是一个分布式系统, 并为该系统
定义安全属性或不变关系。 在分析过程中, 通过判断安全状态或不变关系是否能得到满足来判断安全目标
是否能够得到实现。 模型检测方法具有高度自动化、可以针对漏洞自动生成攻击实例的特点, 在形式化分
析领域取得了很大的成功。


\textbf{通信进程演算方法}

在通信顺序进程(communicating sequential processes, CSP)方法中, 协议的参与者被解释成 CSP 的进
程, 消息被解释成事件, 协议被解释成 CSP 进程的集合。 这些进程并行运行, 并且在运行过程中与它们的
环境进行交互。 对安全协议的验证就是从协议说明中抽取一个模型, 然后使用故障发散提炼(failures
divergence refinement, FDR)来检测协议是否存在问题。

\textbf{Brutus模型检测技术}

Brutus 模型检测技术采用自己的数学模型对密码协议进行建模, 基于一阶逻辑描述协议的安全属性,
并采用状态空间搜索和自然推理相结合的方法, 完成对安全属性的验证。 在进行状态空间搜索过程中,
Brutus 采用了两种状态缩减技术: 偏序状态缩减技术和对称状态缩减技术

目前对于协议分析来讲,模型检测已经证明是一条非常成功的途径,它不仅可用于验证各种传统的安全协议,也可用于验证新兴的量子安全协议。这种方法的自动化程度高,验证过程不需要用户参与;而且如果协议有缺陷,能够自动产生反例。但它的缺点也很明显,如容易产生状态空间爆炸,一般需要指定运行实例和主体数量等运行参数。

\subsubsection{定理证明技术}
基于定理证明的形式化分析将“系统满足其规约”这一论断作为逻辑命题,通过一组推理规则,以演绎推理的
方式对该命题开展证明。基于定理证明的验证大部分是以程序逻辑为理论基础的。
定理证明技术是符号化方法中新的研究热点,定理证明试图将模态逻辑技术的简洁性和模型检测技术的彻底性结合起来,避免模态逻辑中不明确不完善和模型检测中的空间爆炸问题。其将协议的证明规约到证明一些循环不变式中,与证明程序正确性的过程一样。此方面的工作开始于1994年Kemmerer的ITP研究,他通过传统形式规范语言描述安全协议。与Millen\cite{1702206}类似的是,他将协议规范为状态机,但是他的系统中附着了一个证明器,可以把一些性质描述为系统的不变量,证明他们是否被系统保持不变。定理证明方法中最具代表性的是Paulson的归纳证明法\cite{Paulson1997a},他将归纳法应用于协议安全性证明,把协议定义为所有可能事件执行迹的集合,每个执行序列反映了协议主体之间可能的通信方式。Thayer,Herzog和Guttman等人提出了串空间模型(strand spaces)\cite{Fabrega1999}是用图的形式表达协议的执行过程,协议的一个丛就是协议的一个并发运行,协议的安全性质通过所有丛保持的性质来刻画。定理证明方法的一个优越性是可以处理不受限会话数目和不受限消息大小,典型的定理证明工具有HOL公理证明器\cite{Brackin1998}、Isabelle 定理证明系统\cite{Bella1997}等。基于证明的协议验证方法是今后协议分析的一个主要发展方向,其主要优点是语义的精确性和证明推导的完备性,主要缺点是证明过程相当复杂,证明时需要一定的经验,验证速度较慢,且难以完全自动化,大多数方法当协议有缺陷时无法进行错误定位,如果证明针对的是有限的协议运行实例,也只能证明在假设条件下协议是否正确。

符号方法通过Dolev-Yao模型中的完善密码假设使得协议模型中弱化了对密码算法的研究,很大程度上简化了分析过程。研究人员在此基础上设计了多种自动分析工具,实现对协议安全性的自动分析。但也正是对现实协议和敌手进行了抽象,符号方法忽略了许多真实可行的敌手的攻击能力,导致很多协议被符号方法证明安全后又被发现存在漏洞。

%\subsubsection{基于状态机的方法}


\subsection{计算方法}
不同于Dolev-Yao 模型的符号方法,另一种基于计算的安全证明方法也逐步受到研究人员的重视。计算方法的基本思想是
将待验证协议$S$的安全性归约到密码算法$P$的安全性,即假设存在破坏$S$
安全性的敌手,尝试构造破坏 $P$安全性的多项式时间算法;如果$P$
的安全性难以攻破,那么$S$的安全性成立。由于这种方法依赖于计算
复杂度,因此被称为计算方法,相应的分析模型被称为计算模型。

其中Bellare 和Rogaway在1993首先提出了一种名为随机预言机的方法(random oracles)\cite{Bellare1994},其主要思想与密码算法的可证安全的方法类似:假定协议各方同时拥有一个公开的随机预言,就可以在密码理论和应用之间架起一座“桥梁”。但是这并非是严格意义上的安全性证明,因为协议的安全性证明仅仅在随机预言模型下成立,证明中的“取代”过程本质上是一种推测。对于每个新的协议都必须从头至尾地重新证明。Bellare等人1998年利用模拟的思想又设计了模式(modular)方法\cite{Larvala}。计算方法的基础是计算模型,其中消息用比特串表示.协议和攻击者被看作是随机的概率多项式时间图灵机。计算模型更接近于协议的实际运行情况,被认为是目前较为可靠的安全分析手段,但证明难度大且不易实现自动化。

\subsection{安全协议形式化分析方法的进一步发展}
随着研究者对形式化方法认识的深入,对上述两类方法的使用条件及应用范围有了清醒的认识,符号方法与计算方法在对安全协议分析的过程中各有优劣。因此,如何统一这两种方法的优点成为了新的研究热点。
\subsubsection{符号方法与计算方法的结合}
符号方法
通过完善的密码算法假设巧妙的避开了密码算法带来的各种概率和
计算复杂度问题,使得分析过程大为简化。研究人员在此基础上设计
了多种自动分析工具
,实现对协议安全性的自动分析。但也正
是对真实协议和敌手进行了抽象, 符号方法忽略了许多真实生活中敌
手的攻击能力,导致很多协议被符号方法证明安全后又被发现存在漏
洞。而计算方法由于没有对敌手能力和消息形式进行抽象,其分析模
型与真实世界非常一致,被认为是目前最可靠的安全分析手段(尽管
学界颇有微词
) 。然而由于需明确考虑计算复杂度和概率问题,计
算方法的分析过程更加复杂,难度很大,且非常容易出错。目前大多
对安全协议的计算方法分析过程都只能手工分析, 且分析结论存在问
题。

由于分析模型不同,符号方法和计算方法各有优缺点。对上述两种主流形式化分析方法,直观的认识是: 基于计算复杂性的方法是(密码学)可靠的。而(大多数)建立于Dolev-Yao模型之上的形式化方法没有真正建立起密码学的可靠性。
为了综合符号方法和计算方法的优点,得到既可靠又易自动化的
方法,研究人员提出了计算可靠性方法、可模拟性方法和计算模型下
的直接证明法。计算可靠性的方法以符号模型为基础,旨在建立一系
列充分条件,证明在这些条件下符号模型下的安全性蕴含计算模型下
的安全性。
典型方法有 AR逻辑及其扩充,、安全信息流的计算可
靠性、迹映射方法和观测等价方法的可靠性。可模拟性的方法对符号模型和计算模型下协议的执行过程进行关联研究,指出如果计
算模型下协议的执行过程与符号模型下协议的执行过程不可区分,那
么计算模型下的协议与符号模型下的协议具有相同的安全性,Abadi和Rogaway定义了加密表达式的简单语言\cite{Abadi2007},证明了如果两个表达式在逻辑公式下等价,则它们在计算的解释下,根据计算不可区分的标准概念是等价的。
计算模型下的直接证明法以计算模型为基础,旨在研究计算模型下协议执行流程、敌手能力和密码
算法的直接建模方法,通过基于博弈序列变形的推理方法,直接验证
计算模型下协议的安全性,例如计算模型协议组合逻辑 CPCL
和基于博弈序列的方法。

\subsubsection{敌手模型拓展}
符号方法中的Dolev-Yao敌手模型模型具有三大重要假设:
(1)敌手对网络通信具有完全的控制能力,即敌手可对网络通信进行阻断、篡改、注入和重定向攻击;攻击者的能力包括: 熟悉加解密算法、知道参与协议的实体及其公钥、拥有自己的加解密密钥、对网络有完全控制能力、可以用其拥有的密钥进行加解密操作、插入新消息、重放任何消息以及生成随机数。
%Dolev-Yao 模型也存在部分缺点, 如无法针对安全漏洞进行分析, 以及将攻击者能力设定过强导致的拒绝服务攻击使得电子商务协议的匿名性无法得到保证。(此处要进行修改)
(2)完善的密码假设,该模型假定安全协议应用的密码学算法均为安全、不可破解的,即敌手无法破解密码算法,例如对称加密、非对称加密以及哈希函数都是完美其设定的功能而不存在安全缺陷。
(3)协议消息为符号项,即协议消息只能按预定的规则计算,不能随意分解。

Dolev-Yao敌手模型基于完善密码假设,忽略了敌手在真实环境下的对密码学可靠性的攻击能力。在安全协议的实际分析中,这些被忽略的细节会造成新的问题,使得已经被传统敌手模型分析证明是安全的协议在日后也被发现存在漏洞。因此,需要拓展传统敌手模型,对敌手在真实环境下的攻击能力进行建模,以完善分析过程,使得证明的安全性更加可靠,或对一些已经证明安全的协议找到以前未曾发现的漏洞。

当前很多人工分析出的协议漏洞都与敌手对消息的猜解能力有关,这意味着在符号方法的自动化分析中Dolev-Yao敌手模型无法找到所有的协议漏洞。研究者的解决思路是建立符号敌手模型和计算敌手模型的某种联系,以使得符号模型中的安全属性能对应到计算模型中的安全属性。尽管已经有一些相应的成果,但是目前这个方向仍存在两个问题:一是结合后的模型太过于复杂,无法通过简单的Dolev-Yao模型进行推理;二是这些成果在密码系统中能处理的类型太少。例如Backes提出的模拟框架\cite{backes2007reactive},已被证明无法拓展应用于单向哈希函数\cite{Backes2004}和异或运算\cite{Backes2011}。这证明在敌手模型拓展领域仍有许多需要工作需要开展。

在符号方法中,如果验证方法的执行模型或敌手模型不能准确描述真实协议的执行过程或敌手攻击能力,即使协议被证明安全,但仍然存在可被攻击的漏洞。安全协议在面对不同能力的敌手时应当具有较强的适应性,因此对真实的敌手能力进行描述,一直是安全协议形式化研究工作的重点。大多数符号方法是基于Dolev-Yao模型的,这意味这协议参与者或是诚实用户并将一直保留自己的秘密消息或者是彻底恶意用户并将一直受到敌手控制,无法代表敌手真实的攻击能力。当前主流形式化分析方法很少考虑敌手对密码算法的破译能力
(例如对短期密钥和短明文的猜解)和对协议参与节点的攻陷能力,导致很多符号方法验证为安全的协议仍存在漏洞(POODLE攻击\cite{Moller2014}、 Lucky13攻击\cite{AlFardan2013})。 而计算方法中,研究人员通常会建立更贴近实际的敌手模型\cite{Dowling2015},例如在协议运行中攻陷诚实用户变为恶意用户\cite{Backes2004}。然而,这些更具体的模型也有其缺陷,即就是这些模型只是为了密钥协商协议而定义的,并没有拓展适用于一般的安全协议,与此同时,很多现有的自动验证工具并不支持这种强敌手模型。因此,研究者亟需更深入地研究敌手模型。针对Dolev-Yao 模型的不足,研究人员提出了CK模型、eCK 模型\cite{Canetti2002}等敌手模型,eCK模型中的敌手除了拥有Dolev-Yao 能力外,还能获取任意协议主体的长期私钥、临时密钥和特定的会话密钥等。但它主要用于认证密钥分发协议,仍不能完全满足研究人员对安全协议分析的需要。在安全协议分析中,需要对敌手模型进行更精确的描述,才能够更加准确地分析协议的安全性。


%同时部分协议分析者还注意到协议复合性的问题,为了解决多方协议的复合问题,Canetti在2001年提出了通用可组合方法\cite{Canetti2001},它可以用一般的方法描述多方协议的复合。这个框架对于不同的协议用相同的方法处理安全的概念。其中协议与相关的理想协议进行比较,如果实际协议可以在任何环境对于任何攻击者都有与理想协议同样的“行为”时,就认为这是协议的一个安全的实现。


\subsection{安全协议自动验证工具}
研究人员设计了多款主流的形式化工具,如符号模型下的AVISPA\cite{Armando2005b}、ProVerif\cite{Blanchet2001}、Maude-NPA\cite{Escobar2009}、Scyther\cite{Cremers2008}、Tamarin\cite{Meier2013b}等,计算模型下的CryptoVerif\cite{blanchet2007cryptoverif}等。工具简要介绍如下:

AVISPA的设计初衷是为解决大型网络安全协议的描述和验证。它提供了模块化的形式化语言并整合了四种后台验证器\cite{Armando2005b},不仅可以发现协议攻击和漏洞,还可以对有限和无限数量会话的协议进行验证。研究者利用AVISPA 工具对IKEv2-DS协议进行分析并发现了新的攻击\cite{Armando2005c}。

ProVerif是由Bruno Blanchet开发的形式化分析工具。该工具基于逻辑编程语言Prolog 规则和一种高效的新算法,可以基于一些规则推断事件是否会发生,用以实现协议机密性的验证,并且这种结合使用解决了状态爆炸问题,使得ProVerif支持无限轮数的协议执行。ProVerif提出之后得到了广泛应用\cite{Cheval2013a}\cite{Kusters2009a}。

Maude-NPA是 Catherine Meadows团队开发的,它是NRL工具的升级版,该工具支持多种代数性质:异或运算、加解密运算、运算结合律以及Diffie-Hellman 幂运算,并且支持无限会话模型,不仅可以实现攻击和漏洞搜索,还可以实现安全性证明。该工具内置的代数性质使协议的验证更加方便准确,因此该工具也有着较广泛的应用\cite{Escobar2011}。但是Maude-NPA 工具存在攻击效率较低,并且攻击搜索结果很大程度依赖安全属性的形式化描述方式,相比其他工具自动化不高的缺点。

Scyther是Cremers开发的协议形式化分析工具。该工具对于无限会话以及无限状态集合的协议可以给出明确的终止,并且支持多协议的并行分析。Scyther系列工具基于模型改进算法,对状态集合轨迹有着明确的描述,对攻击搜索、角色的执行以及安全性证明有着巨大的帮助。Scyther系列工具的应用特点包括: 可以输出协议角色的执行轨迹; 可以通过敌手能力组合方便地选择包括eCK 等强安全模型在内的安全模型; 支持协议的并行分析。研究者利用Scyther对IKE协议的分析,发现了IKEv1和IKEv2各个子协议之间存在并行攻击\cite{Cremers2008}\cite{Raza2012a}。

Tamarin是由Meier提出的,该工具利用了Scyther工具的逆向搜索算法,提供两种运行方式: 全自动分析和人机交互模式。Tamarin基于多重集合改写规则描述协议流程,利用一阶逻辑量化协议消息和时间节点,从而实现对协议属性的描述,支持具有复杂控制流的协议或者具有非单调易变全局状态的协议,支持eCK等强安全模型,支持多种代数性质: 加解密运算、运算结合律、Diffie-Hellman幂运算以及双线性对运算。

CryptoVerif建立在计算模型上,协议以概率多项式时间的进程代数表示,消息用位串表示,加密算法表示为位串的函数,通过对特定事件的到达和参数的询问来证明安全属性。CryptoVerif不仅可以验证协议,也可以验证密码算法,它能对双方和三方协议进行分析,Meng利用CrytoVerif实现了对可否认认证协议的自动验证\cite{Meng2011}。但是CryptoVerif的缺点也很明显:对涉及群组协议的分析能力不足;只能对秘密性和认证性进行验证,不能对匿名性、不可否认性和公平性等安全属性进行验证。
综上所述,在工具使用方面,Dolev-Yao 模型下的简单协议分析可以使用ProVerif和AVISPA,运行效率高,攻击搜索比较准确。但是这些工具总体上存在以下几个问题:
(1)对协议的执行规范描述不精确:如无法描述协议多版本并发执行,除Scyther外对群组协议分析能力不足等;
(2)对协议的代数属性分析能力不足:在协议的解密运算、运算结合律、Diffie-Hellman 幂运算以及XOR 运算等代数性质中,仅有Maude-NPA工具支持XOR 运算性质,仅有Tamarin支持双线性对运算。
(3)敌手模型描述不足:仅有Scyther和Tamarin工具支持eCK强敌手模型,而且eCK强敌手模型也存在不足,并未完全描述敌手攻击能力。


在安全协议发布后的实际使用过程中,仍存在违反协议声明安全属性的情况\cite{toolsverify}。经过了30多年的发展,研究人员提出了大量的形式化分析方法,
设计了多种自动分析工具, 但很多方法和工具仅能分析小型的安全协
议(如 Needham-Schroeder、Woolam 协议等) 。对于大型的协议(如 IPSec、SSL/TLS等),由于其运行方式复杂,这
些自动分析工具仅能对其部分子协议进行初步分析, 难以给出全面可
靠的分析结论。即使某些安全协议被多种方法多次证明安全,但后来
也被发现存在安全漏洞。对实际部署的复杂协议进行可靠的安全性分
析,是当前的研究热点,也是安全协议形式化分析技术的落脚点。

\subsection{协议实现层的形式化验证}

协议实现层的形式化验证也越来越受到研究者的重视。
由于在形式化方法对协议建模的过程中不可避免地会对实际协议运行细节进行抽象以及简化,这就造成了建模复杂度与协议实际复杂度的不均衡,同时,越是复杂的协议就越难对协议细节进行建模,在一些特殊条件下,就会造成已被形式化方法证明的协议在实际运行过程中仍然可能出现漏洞的情况。以SSL/TLS 协议安全为例,自1994 年被设计出开始,SSL/TLS 协议经历了SSL2.0、SSL3.0、TLS1.0、TLS1.1、TLS1.2、TLS1.3共6个版本变迁,基本上每个版本都有形式化方法证明安全,但随后又发现存在漏洞。TLS1.0 协议曾在符号模型和计算模型下被证明安全\cite{Meyer2013}\cite{Ray2009},但基于浏览器的SSL/TLS 攻击(BEAST)\cite{Duong2011}和RC4 bias 攻击\cite{AlFardan2013b}可利用TLS 的分组加密和RC4 流算法的缺陷发起攻击;TLS1.2 被T.Jager 等人在标准计算模型下证明安全\cite{Jager2012},但压缩算法信息泄露攻击(CRIME)、时序信息泄露攻击(TIME)\cite{Malhotra2016}和Lucky13\cite{AlFardan2013}可利用协议实体计算产生的漏洞发起攻击;降级加密填充预言机攻击 (POODLE)\cite{Moller2014}可利用TLS 版本兼容处理的漏洞发起攻击;重协商攻击可利用TLS 密钥的重协商机制发起攻击。分析这些攻击方法,可以发现现有形式化分析方法中缺少对目标协议的协议多重并发执行、子协议组合、多版本执行等协议执行描述,导致在分析结果中无法定位这些类型的攻击。这对协议安全性分析带来了很大影响。

尽管处于不同的抽象级别,形式化方法中的符号和计算方法都提供了基于抽象模型的严格证明,这些模型与实际应用中的协议实现之间仍然存在很大差距。 即使为该协议模型开发了正确性证明,该差距也可能导致最终无法令人满意的安全级别。 造成这种差距的一个重要因素是开发证明的抽象协议模型与以编程语言编写的实现协议的实际代码之间通常存在很大差异。 例如,协议实现的实际控制流和数据类型通常比抽象模型复杂。 此外,当从模型或规范派生实现时,协议代码中可能会包含逻辑和编码错误,并且这些错误可能无法通过安全测试检测到,从而使协议实现的行为与模型或规范不对应。 现实情况中,由于底层实现的错误,安全协议的软件实现经常需要升级安全补丁。函数调用者错误地解释了函数返回的错误情况,从而使应用程序接受恶意数据。比如在OpenSSL 1.1.1d中,TLS 1.3握手期间或之后调用$SSL\_check\_chain()$函数的服务器或客户端应用程序可能会由于对NULL指针的取消引用而导致崩溃,这是由于对TLS扩展$signature\_algorithms\_cert$处理不正确导致的\cite{cve2020}。

目前对协议实现的安全性分析采用普通软件的方法结合协议软件的特性,主流采用测试的方式对安全协议实现进行测试,对程序进行模糊测试通常要求准确地构造出能触发漏洞的测试用例数据包,以求达到覆盖率尽可能高,测试深度尽可能深的目的。然而,这种技术通常可以用来找到普通的程序错误,如缓冲区溢出漏洞或内存二次释放问题,却无法发现程序中的逻辑问题在实际的测试中,多数据包的输入导致输入空间包含的测试集非常大,不可能对所有可能的输入进行测试。

在对协议实现的研究过程中,研究者逐渐认识到要想系统地分析协议实现在设计逻辑上安全性,需要对协议实现建立严谨的数学的模型,并对此模型进行形式化的分析,才有可能达到目的。如果分析了一个与实现代码没有任何关系的形式化模型,就不会发现这种错误。而形式化方法能够通过抽象的方法,建立协议实现的模型,对其进行形式化分析,规约相同的路径或系统状态,基于规约或推导的方式,将安全属性建模为可解决的问题并进行验证,提供系统的安全或不安全结论。




研究人员首先尝试直接对协议实现进行模型检测,但是Jhala等人证明由于状态爆炸问题,这种穷举的模型检测方式在实际操作中是不可行的\cite{jhala2009software}。直接在源代码上直接证明安全属性的简单方法由于其复杂性而并不实用。 一种常见的实用方法是对协议实现的抽象模型进行形式化分析,并证明协议实现与抽象模型之间存在形式上的联系,通过对模型的分析,以得出有关该协议实现的分析结论。 例如在某些假设下该模型是安全的,那么在相应的假设下该协议实现也是安全的。
这种方法必不可少的两个条件是:正式的抽象建模语言和协议实现,能够通过形式上的语义,对抽象模型和协议实现之间的关系进行明确的定义并给出有效的证明,两个条件缺一不可。 这样即使无法直接对协议实现进行形式化分析,通过验证的抽象模型的安全性也可近似于协议实现。 

提取的模型和协议实现可以通过两种主要方法建立联系,即自动模型提取技术和代码自动生成技术。 这些方法使用的典型工作流程如图\ref{fig-modelextraction},\ref{fig-codegene}所示。

\begin{figure}[htp]
\centering\includegraphics[width=13cm]{data/content/chap01/modelextraction}
\caption{模型提取技术示意图}
\label{fig-modelextraction}
\end{figure}

\begin{figure}[htp]
\centering\includegraphics[width=13cm]{data/content/chap01/codegene}
\caption{代码自动生成技术示意图}
\label{fig-codegene}
\end{figure}


通过模型提取能从带有语义信息的协议实现代码中构建抽象模型,这种抽象将弱化与证明安全属性无关的细节。代码自动生成技术通过协议模型驱动生成代码,并提供抽象模型中缺少的协议运行细节。

这两种方法各有利弊。模型提取可以在老版本的协议实现上运行,而代码自动生成无法实现这一点。 
模型提取的另一个优点是可以隐藏抽象模型,通过合适的组合,用户可以将模型提取和验证工具视为黑盒,直接获取验证结果而不用了解中间步骤。
代码自动生成的优势在于可以通过合理构造生成的代码以避免由侧信道造成信息泄漏(如时序和内存占用)或底层编程错误,例如可能缓冲区溢出等错误。实现构造即正确的理念。

除了模型提取和代码生成方法之外,还存在一些协议实现的安全性分析技术。例如,研究者通过手动编写协议实现代码并建立抽象模型,然后证明抽象模型满足某些属性,并且协议实现是已验证模型的正确实现\cite{abdulla2018model}。
还有的研究者通过编写协议实现的中间抽象版本,一方面运行抽象编译器以导出完全抽象的模型,然后在该模型上进行形式化验证。 另一方面,通过特定的编译器将中间版本转换为协议的具体实现。 这种方法处于模型提取和代码生成之间的中间位置,并同时应用了两种方法的技术。

目前,通过模型提取技术对协议实现的安全性进行分析已经取得了较多成果。
Brumley等人通过谓词公式对实现建模并用于错误搜索\cite{brumley2007towards}。Hsc等人通过提取微软MSN直接消息协议的状态机模型指导模糊测试,发现了许多未知漏洞并成功触发了多次系统崩溃\cite{Hsu2008a}。
Harris等人\cite{harris2009security}对IP网络语音
电话中的会话启动协议实现建立了状态机模型并通过分析其中状态,发现了一个未知漏洞。GUO等人在物联网的IPSec协议中采用模型学习技术,推断了三种IPSec协议实现的状态机模型并对其发现了协议的服务器异常\cite{8913552}。

目前最有效的解决办法是对协议实现建立状态机模型,通过主动学习算法对协议实现进行状态机提取,指导后续安全性分析。这个过程中,首先需要预测协议实现的状态机模型,其次生成测试序列,再次利用已生成的序列对协议实现进行测试,最后观察其响应是否与预测模型一致来不断修正协议实现的模型。在传统的测试序列生成算法中,测试序列生成步骤复杂,学习状态机所用时间长,同时在实际的测试中测试序列数与状态机的状态数及测试深度呈指数式增长关系,只能对小型、简单的协议实现进行测试。
\subsection{形式化方法对协议标准化工作的推动}
通常协议的设计阶段通常不涉及大量的学术分析,并且该过程具有不断针对新发现的攻击进行迭代修补的特征。由于协议的复杂性,早期的科学分析只考虑了高度简化的密码学内核,一旦研究者开始考虑该协议的更详细的方面,就会发现许多新的攻击\cite{Abdalla2015}。例如在1.3版本之前,TLS协议的标准化是通过总结改进上一版本的协议漏洞并制定新的协议标准,并按照新的标准中的协议规范来实现协议的。在研究人员发现相应攻击后,将为主流TLS实现发布临时补丁或在进行中长期的修复并将其纳入下一版本协议标准中。这种做法导致了在整个协议标准的生命周期内,针对协议的攻击与修复补丁不断出现。
而TLS1.3版本的标准设计过程发生了很大变化,互联网工程任务组在制定TLS1.3标准的过程中,积极征询了研究者的意见并鼓励其对提出的草案进行分析。研究者通过形式化方法对TLS1.3的模型和实现进行了深入的分析。

形式化方法能够在确定协议标准和验证协议的过程中找到安全漏洞。例如,对TLS的验证甚至可能只是建立标准的过程中都可能找到存在的安全漏洞。 Bhargavan等人对TLS 1.2的实现进行了验证\cite{Bhargavan2013},发现了新的警报消息,并据此发现了对TLS协议的三次握手攻击\cite{Bhargavan2014}。 Cremers等人使用Tamarin对TLS 1.3第10版草案的协议规范进行了形式化分析并发现了其安全漏洞\cite{2016Automated},该漏洞能使敌手在握手期间模拟客户端,已在其后的第11版草案中修复。Bhargavan等人使用ProVerif对TLS 1.3的协议规范进行了形式化分析,发现了针对第13版草案的针对0-RTT客户端身份验证的安全漏洞\cite{2017Verified}。Cremers等人使用Tamarin对第21版草案的协议规范进行了形式化分析,找到了一些阻止身份验证的意外行为\cite{cremers2017comprehensive}。 这些发现最终在对协议标准的设计产生了积极的影响。

形式化方法的一大优势是能够利用计算机辅助进行分析和验证,由于协议的标准化过程中不断提出新的草案,自动化证明较人工更容易地进行协议标准化推进。TLS 1.3的设计中,其协议规范每隔一段时间就会进行重大改动。
在总共28个草案的迭代中,针对每个草案的分析通常在几个月内就变得过时了,需要不断对新的草案进行分析和证明。随着协议草案的迭代发展,通过计算机自动化的形式化分析明显优于手工分析。除此之外,形式化方法可以确保在更改协议模块时不会引入新的安全漏洞。协议标准化流程可以通过包含较小的更改来简化分析,这些更改可以简化安全性参数并帮助模块化推理。例如,通过协议规范和协议实现一致的标记方案,能够简化协议的一致性证明。 


\section{本文主要工作及内容安排}
论文主要研究了安全协议形式化分析技术,以协议安全性分析为目标,针对当前形式化分析方法中分析模型不完善,分析层次不系统的问题,提出对安全协议进行规范与实现两个层次上的综合性分析,为协议的安全性分析提供有力支撑。

论文的基本思路是:首先,针对敌手能力建模问题,研究基于操作语义的敌手模型拓展;其次,针对递归协议规范安全性难以判定问题,研究基于拓展敌手模型的递归协议安全性判定算法;再次,针对基于安全协议实现的状态机模型提取中存在的状态爆炸问题,研究基于状态机学习的协议实现模型提取技术;最后,协议实现逻辑漏洞难以发现问题,研究基于模型检测的状态机实现分析技术。

全文主要分为七章,其组织结构及各章关系内容之间的相互关系如图\ref{fig-struct}所示。

\begin{figure}[htp]\centering\includegraphics[width=15cm]{data/content/chap01/struct-2}
\caption{文章结构}
\label{fig-struct}
\end{figure}

第一章对协议安全性分析的研究背景和意义、研究内容、发展概况及目前有待解决的问题进行论述,然后对国内外相关领域的研究进行了梳理,最后介绍论文的主要内容和结构框架。

第二章介绍了协议形式化分析的基础理论和方法,包括协议规范建模方法和协议安全属性建模方法,为研究协议的安全性形式化分析奠定技术。随后在协议模型的基础上,介绍了协议的敌手模型用以分析协议安全性,并对常用的协议进行了简介。建立了协议常用分析框架并介绍其基本流程,并给出后续各章节采用的分析的主要思路。本章是后续章节研究的基础和保障。

第三章研究基于操作语义的敌手模型拓展,针对传统形式化方法中敌手模型描述能力不足的问题,建立了基于操作语义敌手模型,该模型拓展了传统敌手模型,敌手将密码学上的敌手攻击分解为三个维度,可以通过三个维度组合建立对应的敌手模型。主要内容包括:\textbf{(1)通过拓展敌手推导规则,增强敌手攻击能力,拓展了敌手攻击模型。
}
传统的敌手模型中,协议参与者要么从开始就是诚实的并将一直保有自己的秘密,要么从开始就是完全的恶意用户并一直被敌手控制。这限制了协议中对诚实用户与被攻陷用户的情况分析。为此借鉴计算安全的敌手模型架构,为敌手能力进行拓展,建立了敌手长期密钥攻击模型和敌手短期数据攻击模型。拓展的敌手模型适用于协议强安全性分析。

\textbf{(2)分析传统Dolev-Yao敌手模型中认为安全的协议,验证了在Dolev-Yao模型下安全的HMQV协议在拓展后的敌手模型下是不安全的。}

拓展的敌手模型能够对弱完美传输机密性、密钥泄露和本地状态泄露进行分析。利用该模型,对传统Dolev-Yao敌手模型中认为安全的HQMV协议进行分析,验证了在Dolev-Yao模型下安全的HMQV协议在拓展后的敌手模型下是不安全的。

第四章针对递归协议的递归原语难以形式化描述,导致递归协议安全性难以判定的问题,\textbf{提出了基于拓展敌手模型的递归协议安全性判定算法,该算法能够对执行正常输入输出以及递归测试的协议进行验证,尤其是安全路由协议以及公钥证书路径的验证。}本章通过对递归协议消息项、敌手能力建模,建立递归协议安全性判定约束系统。在拓展的敌手模型条件下,通过简化规则,将协议拓展敌手模型下的可推理项置于约束系统下。给出了基于映射的递归语言的判定过程,并证明是一个NP问题。

第五章研究了基于状态机学习的安全协议实现模型提取技术,本章的主要内容为:\textbf{(1)针对模型提取过程中测试序列生成慢的问题,提出了检查点算法}

通过构造前缀树用来存储之前执行过的查询,使得测试序列生成所用步骤减少,减少了状态机生成时间。

\textbf{(2)针对测试序列生成数量爆炸问题,提出了套接字约简方法。}

根据安全协议自身的安全机制,利用影响协议后续交互的套接字,约简相应生成的测试序列,显著降低了学习过程的等价查询数量。通过改进算法对 TLS 协议实现的状态机进行学习,建立了协议实现的状态机模型。

第六章提出了基于模型检测的协议实现分析技术,\textbf{建立了基于模型检测的协议实现状态机模型分析框架。}该框架以协议实现的状态机为输入,通过模型检测器对安全属性进行验证,将验证结果分为属性满足和不满足。对验证过程中产生反例通过伪反例检查和反例重放两步进行筛选,确保该反例为真实执行。本章基于协议实现声明,建立了安全属性、密钥交换属性以及函数属性3类共12项安全属性,通过对3种TLS实现库的验证结果显示,GnuTLS 3.6.3基本符合协议规范声明的安全属性;OpenSSL 1.0.1g认证属性不满足,存在一个注入漏洞,其余属性满足协议规范;NSS不满足重协商安全性,其余属性满足协议规范。对于不满足的安全属性定义的协议实现,通过反例检测器,剔除伪反例,找到其对应于协议实现库的协议真实执行。将模型学习与模型检查相结合分析两种TLS1.2实现。研究执行了基于状态机学习的目标系统黑盒测试推断状态机模型,并提取TLS相关规范中的属性,使用模型检测技术自动化分析协议状态机模型。通过分析推断状态机揭示系统的内部结构,所得的模型检测结果可以发现协议规范与协议实现之间的差异,以便在未来版本中对其进行改进。实验证明本方法能够有效分析TLS协议实现库的状态机模型,找到协议实现库存在的逻辑漏洞及与规范不一致的缺陷,包括导致服务器异常的错误交互以及某些违反规范的行为在内的协议实现库逻辑漏洞。这证明通过模型检测方法验证状态机模型,可以寻找协议实现库的逻辑缺陷,达到了对协议实现库的安全性分析的目的。

第七章对全文进行总结,并展望了未来工作中值得继续深入研究的问题。 
