%!TEX root = C:\Users\kobix\Documents\gradu\thesis1\thesiscc.tex
\chapter{协议形式化分析基础}
安全协议形式化分析指用数学方法描述和推理,分析安全协议是否满足其声明的安全属性。目前有多种形式化技术可以用来分析安全协议,但是也都有各自的缺陷和可以改进的地方,这就需要进一步分析每种方法的特点,为后续各章提供理论支撑。

本章主要介绍了协议形式化分析的基础理论和方法,介绍了安全协议的敌手模型以及常用分析工具,探讨了常用的协议分析技术,分析了每种技术的原理和在协议安全性分析中的应用。


%\section{安全协议的形式化分析方法}

%最早从事安全协议形式化分析的是 Needham 和Schroeder,二人于1978年提出了安全协议形式化分析的思想,其主要工作是建立了Needham-Schroeder安全协议[1]。如今,在安全协议形式化分析研究领域,主要存在符号化方法和计算方法两大分支:
%\subsection{符号化方法}
%符号化方法的基础是Dolev-Yao模型。Dolev-Yao模型是由Dolev和Yao在1983年为验证安全协议的安全性而建立的[2],该模型对网络的攻击者进行了定义以及行为能力描述。Dolev-Yao 模型是符号化方法中最流行的敌手模型。该模型具有三大重要假设:1)敌手对网络通信具有完全的控制能力,即敌手可对网络通信进行阻断、篡改、注入和重定向攻击;2)完善的密码算法,即敌手无法破解密码算法;3)协议消息为符号项,即协议消息只能按预定的规则计算,不能随意分解。人们几乎是同时开始了一般安全协议分析工具的开发,以便在Dolev-Yao模型的基础上对安全协议进行自动分析。最早用于安全分析的是由Millen开发的工具Interrogator[3],它的原理是企图通过遍历整个状态空间,找到协议安全的漏洞,它的基本思想被后来的许多工具所继承。另一个代表性的工具是NRL分析器[4],它的状态空间是无限的,也成功地应用于许多协议分析场景。随着研究的深入,人们发现Dolev-Yao模型的能力也是有限的,仅可以进行部分漏洞查找,而且也无法在状态变迁过程中加入其他用户。
%1989 年,Burrows、Abadi和Needham 提出了著名的BAN 逻辑[5]。 BAN逻辑通过对协议过程和假设进行形式化,并通过一系列的推理准则进行推理来判断是否能得到预期结果。BAN逻辑具有简洁直观的特点,并且容易理解也易于使用。但同时,BAN逻辑也存在着一些缺点,如对完整的逻辑语义的缺乏,以及无法对新鲜性建立模型。沿着BAN逻辑的方法,许多逻辑被构造了出来,其中大多是BAN逻辑的变种,目的是弥补BAN逻辑的不足。如GNY逻辑[6]、AT逻辑[7]以及SVO逻辑[8]等等。他们大多是基于模态逻辑。然而,逻辑推理的方法有着自己的不足。主要表现为逻辑的抽象性较高,这种抽象性往往会掩盖(或丢掉)协议执行的状态信息,因而难以完全反映协议运行的全貌,有着自己的局限性。
%BAN逻辑之后,许多形式化方法逐渐被应用于对安全协议的分析上。从上个世纪90年代起,安全协议的形式化分析研究出现了空前的繁荣景象,其中一种有代表性的方法就是模型检测方法。模型检测(model checking)是一种强大的有限状态并发系统自动化验证技术,最早用于分析和模拟硬件工作过程。1996年,Gavin Lowe首次将模型检测器FDR用于安全协议验证,结合通信序列进程(CSP)安全协议模型成功发现了针对NSPK协议的攻击[9]。这个结果对于应用模型检测的方法分析安全协议有着典型意义。FDR是应用 CSP开发的模型检测工具,其目的是检测一个通信系统中进程并发运行的性质。对于安全协议的成功应用,激发了模型检测在协议分析领域的热情。不仅如此,这个漏洞的发现使得人们更加清醒地认识到安全协议分析具有的复杂性。基于模型检测的安全协议分析是一种攻击构造方法,又可被称作状态空间搜索方法,其基本思想是将安全协议看作包含一个状态集合、一个动作集合和一个状态转移集合的分布式系统,协议的执行过程即是状态的迁移过程。采用这种方法寻找攻击,“状态空间爆炸问题”是一个主要障碍,由于多个角色、多个会话的相互作用和攻击者行为的不确定性,使得状态的数目会呈指数增长。解决该问题可以从两方面人手,一是控制状态空间大小;二是控制搜索过程。主要做法包括Brutus的偏序规约技术[10]、Murφ的Symmetry状态缩减技术[11]、基于SAT求解器的模型检测技术[12]等。目前对于协议分析来讲,模型检测已经证明是一条非常成功的途径,它不仅可用于验证各种传统的安全协议,也可用于验证新兴的量子安全协议。这种方法的自动化程度高,验证过程不需要用户参与;而且如果协议有缺陷,能够自动产生反例。但它的缺点也很明显,如容易产生状态空间爆炸,一般需要指定运行实例和角色数量等运行参数。
%定理证明技术是符号化方法中新的研究热点,定理证明试图将模态逻辑技术的简洁性和模型检测技术的彻底性结合起来,避免模态逻辑中不明确不完善和模型检测中的空间爆炸问题。其将协议的证明规约到证明一些循环不变式中,与证明程序正确性的过程一样。此方面的工作开始于1994年Kemmerer的ITP研究[13],他通过传统形式规范语言描述安全协议。与Millen[3]类似的是,他将协议规范为状态机,但是他的系统中附着了一个证明器,可以把一些性质描述为系统的不变量,证明他们是否被系统保持不变。定理证明方法中最具代表性的是Paulson的归纳证明法[14],他将归纳法应用于协议安全性证明,把协议定义为所有可能事件执行迹的集合,每个执行序列反映了协议角色之间可能的通信方式。Thayer,Herzog和Guttman等人提出了串空间模型(strand spaces)[15] 是用图的形式表达协议的执行过程,协议的一个丛就是协议的一个并发运行,协议的安全性质通过所有丛保持的性质来刻画。定理证明方法的一个优越性是可以处理不受限会话数目和不受限消息大小,典型的定理证明工具有公理证明器HOL[16]、Isabelle 定理证明系统[17]等。基于证明的协议验证方法是今后协议分析的一个主要发展方向,其主要优点是语义的精确性和证明推导的完备性,主要缺点是证明过程相当复杂,证明时需要一定的经验,验证速度较慢,且难以完全自动化,大多数方法当协议有缺陷时无法进行错误定位,如果证明针对的是有限的协议运行实例,也只能证明在假设条件下协议是否正确。
%符号方法通过Dolev-Yao模型中的完善密码假设使得协议模型中弱化了对密码算法的研究,很大程度上简化了分析过程。研究人员在此基础上设计了多种自动分析工具,实现对协议安全性的自动分析。但也正是对现实协议和敌手进行了抽象,符号方法忽略了许多真实可行的敌手的攻击能力,导致很多协议被符号方法证明安全后又被发现存在漏洞。
%\subsection{计算方法}
%不同于Dolev-Yao 模型的符号方法,另一种基于计算的安全证明方法也逐步受到研究人员的重视,其中Bellare 和Rogaway在1993首先提出了一种名为随机语言机的方法(random oracles),其主要思想与密码算法的可证安全的方法类似:假定协议各方同时拥有一个公开的随机预言,就可以在密码理论和应用之间架起一座“桥梁”。但是这并非是严格意义上的安全性证明,因为协议的安全性证明仅仅在随机预言模型下成立,证明中的“取代”过程本质上是一种推测。对于每个新的协议都必须从头至尾地重新证明[18]。Bellare等人1998年利用模拟的思想又设计了模式(modular)方法[19]。计算方法的基础是计算模型,其中消息用比特串表示.协议和攻击者被看作是随机的概率多项式时间图灵机。计算模型更接近于协议的实际运行情况,被认为是目前较为可靠的安全分析手段,但证明难度大且不易实现自动化。
%\subsection{形式化方法的进一步发展}
%针对上述两种主流形式化分析方法,直观的认识是: 基于计算复杂性的方法是(密码学)可靠的。而(大多数)建立于Dolev-Yao模型之上的形式化方法没有真正建立起密码学的可靠性。因此,如何统一这两种方法的优点成为了新的研究热点。近年来,人们在扩充Dolev-Yao模型上以及证明形式化方法的密码学可靠性的研究上得到了不少成果。Abadi和Rogaway定义了加密表达式的简单语言,证明了如果两个表达式在逻辑公式下等价,则它们在计算的解释下,根据计算不可区分的标准概念是等价的[20]。同时大多数协议分析者都注意到协议复合性的问题,为了解决多方协议的复合问题,Canetti在2001年提出了通用可组合方法(UC),它可以用一般的方法描述多方协议的复合。这个框架对于不同的协议用相同的方法处理安全的概念。其中协议与相关的理想协议进行比较,如果实际协议可以在任何环境对于任何攻击者都有与理想协议同样的“行为”时,就认为这是协议的一个安全的实现 [21-23]。




\section{协议建模方法介绍}

%×
安全协议是在不可靠或者敌意的通信环境下,
保障一定安全特性的分布式算法。在这种环境中,
敌手对于协议可以有各种各样的攻击手段,包括窃
听、 篡改、 截断协议的通信 ,甚至加入任何可能的消
息 ,把协议的消息转向到其它接收者等。因此 ,对于协议的攻击隐晦而多样。这些因素决定了协议安
全性分析的复杂性。

在形式化方法出现前 ,几乎没有一般可以遵循
的安全协议分析的过程。即使在形式化方法出现之
后,许多协议也是依然通过观察的方式,直观地说明
它们的安全性。所谓形式化方法是指用数学方法为基础对目标系统进行规约、构建和验证,直观的说,就是规范语
言和形式推理,在技术上通过精确的数学手段和强
大的分析工具得到支持。其表现形式通常有逻辑、 模型检测、 状态机等等。
规范语言包括语法、 语义以及满足关系等几部分。
规范语言可以分为四类:抽象模
型规范法、 代数规范法、 状态迁移规范法和公理规范
法。这些方法都可用来
规范 (描述)以及分析安全协议。
本研究主要应用状态迁移规范法对协议模型进行描述。
%×

\textbf{协议规范}

协议规范是协议行为的静态描述。安全协议的协议规范应定义协议的基本项集Term,如新鲜数,协议参与方,变量等;
%其次描述协议的项交互关系(Operator),如拆分算子,配对算子等;最后将角色(Role)定义协议行为的集合,它由协议事件构成,事件之间存在序关系,因此协议的序关系就可以表示为协议的角色事件序关系。协议规范就是将角色映射到角色事件的部分函数。

\textbf{协议执行描述}

协议执行描述能够表示协议的动态特征。这些特征包括协议的节点模型以及执行模型。
%执行描述首先定义了协议的运行项(RunTerm),它与基本项类似,不同的是运行项有自己的运行标识。其次,为了能够将协议消息等价为受角色项规范的样式,定义匹配算子(Match),这样就能对协议消息的进行描述。最后就能够通过运行事件(RunEvent)将协议描述(如项,事件)与其执行联系在一起。

\textbf{安全属性建模}


安全属性作为安全协议的一个基本属性,不能离开了协议的安全目标而独立存在。因此对安全属性的建模应该与协议规范相一致。可通过安全断言来结合安全属性与协议规范。在协议规范中定义谓词诚实(honest),定义行动函数(Actor)以表示节点的特定运行事件。安全属性建模应满足多种协议模型下的统一,可以通过逻辑推导,定理证明方法中目标属性的等价性,混合验证目标属性的可满足性。通常协议在满足一条安全属性的同时还可能满足多条安全属性,结合协议组合逻辑,可以通过包含不同安全属性的子协议复合而成。安全协议的安全属性通常包括以下四类:
\begin{enumerate}
\item 
机密性表示即使数据在不可信的网络中传递,敌手也不能获得其中的信息。于是机密性断言可以通过操作语义表示:对角色中任何映射到诚实节点的运行,如果项不能被敌手的知识获得,那么可以断言该项的机密性可满足。
\item 
认证性有多种形式。最基本的形式为:认证性是一个通信对象存在的简单声明。一个协议正常运行至少需要两个节点进行交互。可以定义多种形式的认证性,如存活性,一致性等。
存活性的定义为,如果一个目标交互对象完成了某些事件,那么称这个通信对象是存活的。这时的存活性断言也可通过操作语义表示:对执行了断言事件的节点,如果目标交互对象诚实,那么他会执行一个事件。
一致性的定义为接收消息的内容与发送消息的内容一致。一致性断言的操作语义表示为:对协议中的所有断言,存在其他角色的运行,这样所有在断言之前的交互事件必须在该断言之前完成。
\item 完整性是指信息不被偶然或蓄意地删除、修改、伪造、乱序、重放和插入等特性。非授权写是对完整性的破坏。
  完整性的目的是保护协议消息不被非法改变,删除和替代。最常用的方法是封装和签名,用签名或者Hash产生一个消息的摘要附在传送的消息上,作为验证消息完整性的一个依据,称为完整性校验值。一个关键性的问题是,通信双方必须实现达成有关算法的选择等款项的共识。
\item 不可否认性包括收发双方均不可否认(抵赖)已经发生的事实。一是源发证明,它提供给信息接收者以证据,这将使发送者谎称未发送过这些信息或者否认它的内容的企图不能得逞;二是交付证明,它提供给信息发送者以证据,这将使接收者谎称未收到这些信息或者否认它的内容的企图不能得逞。

\end{enumerate}


%%%\subsubsection{认证性}
%%认证性有多种形式。最基本的形式为:认证性是一个通信对象存在的简单声明。一个协议正常运行至少需要两个节点进行交互。可以定义多种形式的认证性,如存活性,一致性等。
%%存活性的定义为,如果一个目标交互对象完成了某些事件,那么称这个通信对象是存活的。这时的存活性断言也可通过操作语义表示:对执行了断言事件的节点,如果目标交互对象诚实,那么他会执行一个事件。
%%一致性的定义为接收消息的内容与发送消息的内容一致。一致性断言的操作语义表示为:对协议中的所有断言,存在其他角色的运行,这样所有在断言之前的交互事件必须在该断言之前完成。



\section{敌手模型介绍}
敌手模型是表征攻击者在网络中具备的攻击能力的模型。作为协议形式化分析的重要组成部分,敌手模型是衡量协议安全性的参考标准,协议的安全与否与敌手能力息息相关。敌手的攻击能力的强弱,直接体现在使用的敌手模型之中。传统基于符号的形式化方法采用的敌手模型是Dolev-Yao模型,它限定了网络诚实节点的和恶意节点的边界:节点从始至终是诚实节点或者一开始就是在敌手的控制之下。 在这种局限的观点下,Dolev-Yao模型无法验证密钥交换协议所提供的安全性是否真的安全。在计算方法中,敌手能力得到了极大强化,其中具有代表性的是对网络具有绝对控制能力的eCK敌手模型。%×


\subsection{Dolev-Yao敌手模型介绍}

Dolev-Yao敌手模型是目前符号方法中应用最多的敌手模型。该模型具有三大重要假设:
\begin{itemize}
\item 敌手对网络通信具有完全的控制能力,即敌手可对网络通信进行阻断、篡改、注入和重定向攻击;攻击者的能力包括: 熟悉加解密算法、知道参与协议的实体及其公钥、拥有自己的加解密密钥、对网络有完全控制能力、可以用其拥有的密钥进行加解密操作、可以插入新消息、可以重放任何消息、可以生成随机数。Dolev-Yao 模型也存在部分缺点, 如无法针对安全漏洞进行分析, 以及将攻击者能力设定过强导致的拒绝服务攻击使得电子商务协议的匿名性无法得到保证。
\item 完善的密码假设,该模型假定安全协议应用的密码学算法均为安全、不可破解的,即敌手无法破解密码算法,例如对称加密、非对称加密以及哈希函数都是完美其设定的功能而不存在安全缺陷。
\item 协议消息为符号项,即协议消息只能按预定的规则计算,不能随意分解。
\end{itemize}

\subsection{eCK模型介绍}

eCK模型阐释了如何用密钥交换协议安全模型形式化描述恶意攻击\cite{LaMacchia2007}。本论文第三章中的敌手模型部分出于此模型。

eCK模型假设敌手完全掌控了网络,它能够窃听、重定向、转移任何信息以及按自己意愿任意注入消息。此外,敌手还可以获得3种秘密消息:
\begin{itemize}
\item 攻陷节点以获得其长期密钥;

\item 通过密码学分析或攻陷节点以获得其会话密钥

\item 通过侧信道攻击或者破解弱随机数生成器获得节点产生的随机数
\end{itemize}

eCK模型描述了一个密钥交换下的博弈模型以及胜利条件\cite{Bellare1994}。如果敌手(建模为概率多项式时间图灵机)在博弈过程中无法以不可忽略的概率猜到胜利条件,就证明协议是安全的。简单地说,在博弈过程中,可以与执行协议实例的节点交互,如果它能从一串随机比特序列中识别出一个真实的会话密钥,则敌手获得博弈胜利。

认证密钥交换协议通常包含两种角色:发起者(发起协议会话的节点)和应答者(受到消息并响应的节点)。一个节点扮演一个角色,一个角色执行的角色实例为一个线程。在计算模型中,一个线程往往被视为一个本地会话。用$tid$码来标记识别线程。

\textbf{eCK博弈}

使字母$A、B、C$…来表示角色的身份识别码。每个角色都持有一个与其相关的长期公/私密钥对。起始时,敌手知道每个角色的身份和其公钥。敌手可以执行一系列查询,来获知角色的私钥。在eCK模型中,查询动作如下。

\begin{itemize}
\item $Test(A,B,comm)$. 如果消息$comm$是空消息,这个查询动作表示恶意敌手启动线程。实际上,角色$A$激活一个启动线程试图与B通信,并计算第一条发送用消息。这个查询动作的结果就是恶意敌手获得了这条信息。恶意敌手可以窃听所有的消息。如果消息$comm$不为空,这个动作表示敌手向$A$发送消息$comm$,声称消息来源于$B$。该消息可以被任何活动线程接收,并更新状态,或者让$A$产生一个用于接收消息的新线程。无论是哪种情况,敌手都获得了A的消息。
\item	$LongTermKeyReveal(A)$. 敌手获得了A的长期私钥。
\item	$EphemeralKeyReveal(tid)$. $tid$代表活动线程的身份编码。敌手获得了$tid$的短期密钥。通过这一步行动,敌手可以获得线程之前产生的所有随机值。
\item	$Reveal(tid)$. $tid$是由会话密钥k计算得出的,敌手获得了密钥k。
\item	$Test(tid)$ .该行动仅能进行一次。随机数产生器生成一个随机数$b \in {0, 1}$。如果b=1,敌手获得密钥k。如果b=0,从密钥空间选取一串随机比特序列给敌手。
\item	$Guess(b’)$. 该动作只能在$Test$动作之后进行。该动作之后博弈结束。该动作表示敌手猜测它是否获得真实会话密钥。

\end{itemize}

如果模型中没有加入额外限制,敌手可以通过$Reveal(tid)$和$Test(tid)$猜测得到正确的随机比特b。这相当于敌手可以在任何时候获取所有会话密钥,因此没有协议可以确保会话密钥能在该敌手面前保持秘密。因此,为了确保关于认证密钥交换模型中协议的正确性,这些模型限制了敌手可以进行的查询。eCK模型旨在指定尽可能少的限制,但同时又能保证一些协议是正确的。

\textbf{clean属性}

在eCK模型中,引入了一个属性$clean$来描述敌手已经满足特定线程的预期限制。将敌手执行的一系列查询动作称为$experiment$。
为了对$clean$属性建模,还引入了匹配线程$matching threads$。通常一个密钥交换模型的目的是建立一个两个线程之间的共享密钥。因此,给定一个线程$tid$, 如果线程$tid’$能够计算相同的会话密钥,则称这两个线程为匹配线程。
eCK模型中的双消息协议的匹配线程。设$tid$为一个线程,称$(role, ID, ID^∗,comm1, . . . ,commn)$,为标识元组。$Role$代表线程角色,$ID$是执行线程的角色标识,$ID^∗$是通信参与者标识,$comm1, . . . ,commn$是消息序列。对于一个双信息协议,若$r $不等于$ r’$,且两个线程的标识元组为$(r,A,A_,comm1,comm2), (r',A,A,comm1,comm2),$称这两个线程为匹配线程。



设$E$表示一个$experiment$,$tid$表示由$A$执行的通信线程,A通过$tid$与B进行通信。在E中,若匹配线程$tid^*$存在,当且仅当下列条件成立时,称$tid*$具有$clean$属性。

(1)$E$ 包含了$Reveal(tid) $或 $Reveal(tid∗)$

(2)$E $包含了$ LongTermKeyReveal(A)$ 和 $EphemeralKeyReveal(tid)$

(3)匹配线程$tid^*$存在且$E $包含了$ LongTermKeyReveal(B) $和$EphemeralKeyReveal(tid^*)$

(4)匹配线程$tid^*$不存在且 $E$ 包含了$LongTermKeyReveal(B)$

eCK博弈的胜利条件:若在$experiment(E)$中,敌手得到$b' = b$且Test线程是clean的则敌手获得eCK博弈的胜利。

\textbf{eCK安全}

若在eCK博弈中,匹配线程计算相同的会话密钥,且敌手没有不可忽视的优势。称协议在eCK模型中是安全的。


eCK模型这样的敌手模型很复杂,整体的定义很难梳理,设计理念也很难理解。一般情况下,由于敌手定义、匹配线程定义、执行模型以及安全属性细节方面的差异,两个计算模型之间是很难比较的。研究者们研究了认证密钥交换模型中的一些定义及其关系的细节并得到了一些成果\cite{RaymondChoo2006}\cite{Cremers2013}\cite{LaMacchia2007}。




\section{常用安全协议介绍}
安全协议是建立在密码体制基础上的一种交互通信协议,它运用密码算法和协议逻辑来实现认证和密钥分配等目标。
本节主要介绍三种跟本研究相关的安全协议:HMQV协议、递归协议和TLS协议。其分别对应本文第三章、第四章、第五和第六章研究内容。

\subsection{HMQV协议介绍}
HMQV协议是MQV协议中用于密钥分配的经典协议,在2005年由Krawczyk提出\cite{krawczyk2005hmqv}。
MQV协议族是一类认证密钥交换协议族,曾经是应用Diffie-Hellman公钥技术进行身份验证的协议中效率最高的协议,MQV协议曾被NSA选为“下一代加密技术以保护美国政府信息”的密钥交换机制。首先对其进行介绍,如图\ref{fig-HMQV}所示。
\begin{figure}[htp]
	\centering
	\includegraphics[width=8cm]{data/content/chap03/HQMV}
	\caption{HMQV协议示意图}
	\label{fig-HMQV}
\end{figure}

HMQV协议有两种角色,Initiator和Responder,分别记为A,B。A的长期私钥和公钥分别为$a$和$g^a$,B的长期私钥和公钥分别为$b$和$g^b$ 。协议交互过程如下:

	1.A生成新鲜数$x$并发送$g^x$给B;

	2.B收到$g^x$后,生成新鲜数$y$并发送$g^y$给A;

	3.双方根据交换的值以及自己的长期私钥计算出双方的会话密钥。


将HMQV的设计和证明建立在从Schnorr识别方案派生的新形式的“挑战响应签名”的基础上,该属性具有挑战者和签名者都可以计算相同签名的特性。 前者通过选择挑战,后者通过了解私有签名密钥, 

\subsection{递归协议介绍}
递归协议是一类组协议,在递归协议中参与者在协议执行过程中执行的动作通过$t\to s$形式的单个重写规则来描述,该过程通常涉及复杂的迭代或递归计算,通常称之为递归协议。 其中参与者在每个协议步骤中都可以发送无限制的大量消息\cite{truderung2005selecting}。 这些消息中的每一个都是对到目前为止收到的消息的某些子项应用一些简单的重写规则的结果。
但是,在许多协议中,参与者执行的操作更加复杂,递归计算,无法通过简单的重写规则来表示。 例如,递归身份验证协议(Recursive Authentication,RA)中,节点$A$与节点$B$联系。如果节点B随后与认证服务器联系,则认真服务器将生成会话密钥$K_{ab}$并将其分发给$A$和$B$ 但是$B$可以选择联系其他节点$C$,依此类推;该递归可以形成任意长度的链。 在每轮递归中,代理将其名称和新的随机数添加到不断增长的请求消息中。
从最简单的条件距离,假设$C$不扩展链,而是联系身份验证服务器。 服务器生成新的会话密钥$K_{ab}$和$K_{bc}$:通常情况下,链中相邻的每对代理对都有一个密钥。 它为每个会话密钥准备两个证书:每个参与者一个。 它将证书包提供给最后一个代理$C$。每个节点获取两个证书,并将其余证书转发给链中的前任。 最后,A收到一个包含$K_{ab}$的证书。
这样的递归协议很难分析。在事先没有确定步骤数,参与者数或会话密钥数的情况下,服务器对代理累计请求的响应不能以简单的方式给出,而需要一个递归表示。


\subsection{TLS协议介绍}
TLS的设计目标是在传输层上构建一个安全传输层,用来确保网络通信中的三种主要安全性质:
它能够提供以下三种主要服务:
\begin{itemize}
\item \textbf{认证性},认证用户和服务器,确保数据发送到正确的客户机和服务器;

\item \textbf{机密性},加密数据以防止数据泄漏;

\item \textbf{完整性},维护数据的完整性,确保数据在传输过程中不被改变。
\end{itemize}
\subsubsection{TLS协议结构}
TLS协议结构如图\ref{fig-TLSstructure}所示:

\begin{figure}[htp]
	\centering
	\includegraphics[width=8cm]{data/content/chap04/TLS+6}
	\caption{TLS协议结构}
	\label{fig-TLSstructure}
\end{figure}

\textbf{握手协议}是SSL的核心,也是最关键的部分。在SSL中握手协议的主要作用
是在客户端和服务器端建立安全的数据传输通道。

TLS握手协议的消息传输如图\ref{fig-handshake}:
\begin{figure}[htp]
\centering\includegraphics[width=6cm]{data/content/chap04/handshake1}
\caption{TLS握手协议}
\label{fig-handshake}
\end{figure}

TLS握手协议如图\ref{fig-handshake}所示,它包含五个步骤:
\begin{enumerate}\item  客户端向服务端发送服务器握手消息,这个消息里包括客户端生成的随机数、客户端支持的加密套件等信息。
\item  服务器向客户端发送服务器消息,服务器在客户端支持的密钥组件中里选择一份加密套件,该套件决定了后续加密和生成摘要时采用的算法,另外还会生成一份随机数用于后续认证。同时将服务器的证书发送给客户端验证。
\item  客户端验证服务器证书通过后取出证书中的公钥,将自身证书发送给服务器,通知服务器开始使用加密方式发送报文和计算消息认证码,并对之前握手的消息用主密钥加密后传输。
\item  TLS服务器验证客户端证书后,发送更改密码规范消息,确认使用的密码套件。
\item  通信双方以握手过程中协商好的安全参数,进行安全的应用数据传输。
\end{enumerate}

\textbf{更改密码规范协议}只有一条消息。握手过程协商结束后,通信双方会互相交
换更改密码规范协议消息,用以通知对等端启动握手过程中协商好的各种安全参
数,从而保护后续的消息传输。

\textbf{警告协议}由Close-Notify消息实现。在安全的网络通信中,不管是在协商阶段
还是在数据传输阶段,一旦通信中的一方发现连接或传输出现了差错,都要立即
向另一方发出警告;警告协议的另一种功能是在数据传输完成后,通知通信的另
一方断开连接

\textbf{记录协议}是TLS协议中承载数据的部分,其中TLS的记录格式和对记录的处
理流程都由记录协议规定。应用报文和握手协议处理过的数据都由记录协议封装
成记录报文并传输。




\section{有限状态机介绍}
%%%通过状态机学习技术,可以建立安全协议实现对应的状态机模型。通过对协议状态机的分析,能够将协议实现的安全属性分析问题转化为该状态机的安全属性分析,这有助于对协议实现的状态状态转移关系进行细致的分析,找寻其中对应的逻辑问题,而关注点将不再是常规程序分析中对协议实现的内存问题的寻找。通常协议状态机的状态转移不仅与当前接收到消息有关,也与此时状态机的状态有关。
有限状态机(Finite State Machine,FSM)是表示状态以及在这些状态之间的转移和动作等行为的数学模型,简称状态机,它由状态、转移、事件和动作组成。状态机描述了一个系统在生命周期内所经历的各种状态,状态之间的转移,发生转移的动因、条件以及转移中所执行的活动,其本质是对具有逻辑顺序或时序规律事件的一种描述模型\cite{tanabe2020model}。有限状态机广泛应用于硬件电路系统设计、软件工程、人工智能与网络协议等领域。

状态机有两种类型:确定性有限状态机(Deterministic Finite Automata,DFA)和非确定性有限状态机(Nondeterministic Finite Automata,DFA)。 确定性有限状态机对于一个给定的输入只有一个确定性的输出和状态迁移。非确定有限状态自动机是对每个状态和输入符号对可以有多个可能的下一个状态的有限状态自动机。
状态机包含4个组成要素,即现态、条件、动作、次态。它们的结合能够完整地表示系统的内在状态迁移关系。现态就是指系统当前所处的状态。条件又称为“事件”,当一个条件被满足,将会触发一个动作,或者执行一次状态的迁移。动作即条件满足后执行的动作,动作执行完毕后,可以迁移到新的状态,也可以仍旧保持原状态。次态为条件满足后要迁往的新状态。次态是现态经历条件满足、执行新动作后发生迁移后的。确定性有限状态机的执行逻辑是从起始状态开始,逐步接受输入的条件,并根据给定的转移函数逐步转移至次态并给出输出,如图\ref{fig-chap02-state_machine}。 
\begin{figure}[htp]
	\centering
	\includegraphics[width=6cm]{data/content/chap04/statemachine}
	\caption{有限状态机的执行逻辑}
	\label{fig-chap02-state_machine}
\end{figure}

\section{模型检测介绍}
模型检测的概念最早在1981年由美国的Clarke和Emerson以及法国的Quielle和Sifakis分别提出,他们开发了一套用于判断硬件和软件的设计模型是否满足规范的方法。模型检测的基本思想是根据有限状态机理论, 通过定义状态集和状态迁移函数为待测系统建立模型, 通过算法穷尽所有状态空间并搜索某些特殊状态, 以判断系统是否达到安全预期。 模型检测技术一般将安全协议看作是一个分布式系统, 并为该系统定义安全属性或不变关系。 在分析过程中, 通过判断安全状态或不变关系是否能得到满足来判断安全目标是否能够得到实现。 模型检测方法具有高度自动化、可以针对漏洞自
动生成攻击实例的特点。

\subsection{模型检测原理介绍}
模型检测是一种形式化自动验证技术,即对于一类给定的
有穷状态并发程序(或系统)和表示系统性质(或规范)的某种时序逻辑公式,能
否找到一种或多种算法,它能够判定系统类中的任一给定系统是否满足公式类
中任意给定的一个时序逻辑公式是否满足其设计规约。
模型检测算法的输入包括二部分,分别是待验证系统的模型$M$和系统待检
测性质$\varphi$的描述,如果模型$M$满足性质$\varphi$,则模型检测器输出“true”;否则模型
检测器给出反例说明$M$为何不满足性质$\varphi$(可显示出详细的路径或过程)。图\ref{fig-modelchecker3}为模型检测原理。

\begin{figure}[htp]
	\centering
	\includegraphics[width=13cm]{data/content/chap02/modelchecker}
	\caption{模型检测原理图}
	\label{fig-modelchecker3}
\end{figure}


模型检测使用状态空间搜索的办法来全自动地检验一个有穷状态系统是否满足其设计规范。这类方法的优点在于它有全自动的检测过程且验证速度快、效率高,并且如果一个性质不满足,它能给出这个性质不满足的理由,并指导对系统的改进。

模型检测的瓶颈问题是状态爆炸问题。由于并发系统自身的特性,当系统所包含的可交互和通迅的子系统数目较多时,整个系统的规模可能产生的状态不可避免地会增大。因此当一个系统的并发分量较多的时候,直接对其状态空间进行搜索在实际上是不可行的,这就是所谓的“状态爆炸”问题。如何使用精简方式描述系统,避免因为引入过多变量而引起状态爆炸问题,为系统建立轻量级模型来解决状态爆炸问题是一个值得研究问题。
模型检测应用过程中,需要考虑增加效率和减少内存空间需求,此外,需要研究减少模型本身或验证性质的复杂性,主要方法有不同类型的抽象、程序切片、模型分解、验证性质的分解等。

%\subsection{Kripke结构和时态逻辑}

\subsection{Kripke结构和时态逻辑介绍}
模型检测中系统模型常用Kripke结构表示,系统要满足的规范用时态逻辑描述。

\textbf{Kripke结构}

模型检测中,Kripke结构是一个四元组$M=(S,I,R,L)$,其中$S$是所有状态的集合,是有限集;$I$是初始状态集合;$R\subseteq S\times S$,表示状态迁移关系;$L:S\to 2^{P}$是一个标记函数。
Kripke结构可以用图形象地表示,如图\ref{fig-kripke}所示
\begin{figure}[htp]
  \centering
  \includegraphics[width=10cm]{data/content/chap05/Kripke}
  \caption{Kripke结构的有向图表示}
  \label{fig-kripke}
\end{figure}

用圆圈表示状态;箭头表示状
态迁移关系;始端没有关联状态的
箭头标记初始状态;标记函数通过
在状态中标注原子命题体现,在状
态中标注的原子命题表示在该状态
下,该原子命题成立,否则原子命
题不成立。图\ref{fig-kripke}对应的Kripke结
构是:

$S={s_0,s_1,s_2};I={s_0};R={(s_0,s_1),(s_1,s_2),(s_2,s_0),(s_2,s_2)};L(s_0)={p_1},L(s_1)={p_2},L(s_2)=\oslash$




\textbf{时态逻辑}

时态逻辑(temporal logic)是一种包含有时间概念的逻辑,是在命题逻辑的基
础上加入了时态算子而形成的逻辑。时态逻辑表示系统的各个事件的发生在时间
上的次序。时态逻辑的种类很多,用在模型检测中的时态逻辑一般是线性时态
逻辑(Linear-time Temporal Logic,LTL)和计算树逻辑(computation Tree Logic,
CTL)。在模型检测中,用时态逻辑描述系统所要满足的规范。
LTL将时间建模为状态的序列,任一当前时刻仅存在唯一的可能未来时刻。
状态的序列称为路径。CTL将时间建模为一个树状结构,任一当前时刻可能分
叉为多种可能未来时刻。
LTL中常用的时态算子(也称为联结词)是$X,G,F,U,R,W$,分别表示“下一
个状态"、未来所有状态”、“未来某个状态"、“直到…都"、“直到…才"和“弱
直到”意思。CTL中除了有LTL中的时态算子外,还有量词$A,E$,分别表示“对
所有路径’’和“存在一条路径",通常CTL中不包含尺,从而CTL中的时态算
子(也称联结词)是$AX,EX,AF,EF,AG,EG,AU,EU$,与LTL联结词的一个明
显区别是,每个CTL联结词都是一对符号。

基于Kripke结构,可以考察LTL公式的语义。设Kripke结构$M=(S,I,R,L)$,
$\pi = s_0\to s_1\to s_2 \to \ldots$是$M$中的一条路径,$\pi^i=s_i \to s_{i+1}\to \ldots$,$\phi$是一个LTL
公式,$p$是原子命题。下面给出一些LTL公式的语义解释:

$\pi \models p$当且仅当$p\in L(s_0)$

$\pi \models \neg p$当且仅当$p\in L(s_0)$

$\pi \models \phi_1 \wedge \phi_2 $当且仅当$\pi \models \phi_1且\pi \models \phi_2$

$\pi \models G\phi $当且仅当 $\forall i ,\pi^i \models \phi$

$\pi \models F\phi $当且仅当$\exists i ,\pi^i \models \phi$

$\pi \models X\phi $当且仅当$\pi^i \models \phi$

同理基于Kripke结构,可以考察CTL公式的语义。设Kripke结构
$M=(S,I,R,L),s\in S$,$\phi$是一个CTL公式,$p$是原子命题。下面给出一些CTL
公式的语义解释:

$M,s \models p$当且仅当$p\in L(s)$

$M,s \models AX\phi$当且仅当对所有$s_i,满足s\rightarrow s_1有M,s\models\phi$

$M,s \models EX\phi$当且仅当对某个$s_i,满足s\rightarrow s_1有M,s\models\phi$

$M,s \models AG\phi$当且仅当对所有满足$满足s= s_1的路径=s_1 \to s_2 \to s_3\ldots,以及沿该路径上的所有s_1,有M,s_1\models\phi$

LTL、CTL联结词中,有些联结词可以由另一些联结词等价表示,即存在着
联结词的完备集。LTL联结词的完备集有如下三种:${U,X}$、${R,X}$和${W,X}$。CTL
联结词的完备集是:至少包含${AX,EX}$中之一、${EG,AF,AU}$中之一以及$EU$。
模型检测算法只要能处理联结词的完备集即可。

\subsection{模型检测算法介绍}
模型检测算法根据协议规范使用CTL还是LTL描述,分为CTL模型检测算法和
LTL模型检测算法。相对来说,CTL模型检测算要直观一些。本节详细介绍CTL
模型检测算法和LTL模型检测算法。

\subsubsection{CTL模型检测算法}

Emerson和Clarke于1981年左右证明了CTL公式的不动点特性。CTL模
型检测算法正是基于这一性质建立的,它也称为标记算法,因为直观上看,就是
通过对满足CTL公式的状态进行标记,找出满足CTL公式的所有状态。

已知一个模型$M=(S,I,R,L),s\in S$和一个CTL公式$\phi$,模型检测就是计算
$M,s\models \phi$是否成立。通过计算出在模型$M$下满足$\phi$的所有状态,然后看结果是否
包含有状态s的方法计算$M,s\models \phi$,要比直接计算$M,s\models \phi$要容易一些。下面给
出详细的算法过程:

输入:一个Kripke模型$M=(S,I,R,L)$和一个CTL公式$\phi$。

输出:M上满足$\phi$的所有状态。

首先,对$\phi$进行预处理TRANSLATE($\phi$),TRANSLATE($\phi$)输出与$\phi$等价的
只由CTL公式联结词完备集和命题公式联结词完备集构成的CTL公式。
其次,从$\phi$的最小子公式开始,用$\phi$的被满足的子公式来标记$M$的状态,并
逐步扩展。假定$\varphi$是$\phi$的一个子公式,且满足$\varphi$的所有子公式的状态都已经被标
记。按如下方法确定哪些状态用$\varphi$标记:如果$\varphi$是
\begin{itemize}
\item 永假:则没有状态被标记
\item $p$:若$p\in L(s)$,则用$p$标记$s,p$是原子命题
\item $\varphi_1 \wedge \varphi_2$:若状态$s$已用$\varphi_1$和$\varphi_2$标记,则$\varphi_1 \wedge \varphi_2$标记
\item $\neg \varphi$:若状态$s$还没有用$\varphi$标记,则用$\neg\phi$标记$s$
\item $AF \varphi_1$:
若任何状态$s$用$\varphi_1$标记了,则用$AF\varphi_1$标记它。

重复:用$AF\varphi_1$标记任何状态,如果所有后继状态用$AF\varphi_1$标记,直到不发生改变为止。CTL不动点表明这会终止。

\item $EX\varphi_1$:用$EX \varphi_1$标记任何状态,如果其后继状态之一用$\varphi_1$标记
\end{itemize}
对$\phi$的所有子公式(含$\phi$本身)完成标记后,输出标记为$\phi$的状态。
这个算法的时间复杂度为$D(F·V·(V+E))$,其中$F$是公式中时态算子的个
数,$V$是状态个数,而$E$是状态迁移个数。该算法关于公式的大小为线性的,而
关于模型的大小是二次的。

下面给出CTL模型检测算法的伪代码。


   \begin{lstlisting}[escapechar=\#]


bool #$Model\_Checking(M,s,\varphi)$#
{
    #如果$Satisfy(\varphi)\cap{s}$#不为空,返回真,表示#$M,s\models \phi$#

    否则 返回假,表示#$M,s\models\phi$#不成立。
}

#$Satisfy(\phi)$//返回满足公式$\phi$的所有状态#
{
  #$switch(\phi)${#
  #case 1: return $S$ ;# //公式永真,返回整个状态集
  #case 0: return $\oslash$;#//公式永假,返回空集
  #case p: return ${s\in S | \phi\in L(s)}$;//返回原子命题$p$为真的状态#
  #case $\neg \phi_1$: #return# $S-Satisfy(\phi_1)$;#
  #case $AF \varphi_1$: #return# $Satisfy_AF(\phi_i)$;#
  #case $EX\phi_1:$ #return# $Satisfy_EX(\phi_1)$;#
  }
}

#{$Satisfy\_AF(\phi_i)$#
{
  #声明局部变量集合$X,Y$;#
  #$X = S;$#
  #$Y = Satisfy(\phi);$#
  do {
      #$X = Y;$#
     #$Y = Y \cup pre_\forall (Y);$#
  }#while ($X == Y$)#
  return Y;
}

#$Satisfy\_EX(\phi_1)$#
{
  #声明局部变量集合$X,Y$;#
  #$X = Satisfy(\phi_1)$;#
  #$Y = pre_\exists(X)$;#
}
\end{lstlisting}

可以看出,算法的实现主要依赖于函数,pre是
沿迁移关系向后移动。它们的具体定义是:

$pre_\exists (X)={s\in S|存在s',(s\to s'且s'\in X)}$

$pre_\forall(X)={s\in S|对所有s',(s\to s'蕴含s'\in X)}$

\subsubsection{LTL模型检测算法}
CTL模型检测算法是相当直观的,但其思想不适合LTL。LTL模型检测必须
采用不同的策略。一般LTL算法采取的基本策略如下:
设Kripke结构$M=(S,I,R,L),s\in S$和一个LTL公式$\phi$。要确定是否
$M,s\models \phi$,即沿$M$的从$s$出发的所有路径,$\phi$是否满足。一般,LTL模型检测算
法都遵循以下三步进行。
第一步:为公式$\neg\phi$构造一个自动机,也叫布景(tableau)。关于$\phi$的自动机
称为彳妒。于是,构造$A\neg\phi$。构造出的自动机恰好接收那些使$\neg\phi$成立的运
行,从而得到$\phi$使不成立的运行。

第二步:将自动机$A\neg\phi$和系统的模型$M$结合。结合运算的结果产生一个迁移系统,其路径即是自动机的路径又是系统模型的路径。
第三步:在结合的迁移系统中搜索,看看是否存在由J导出的状态出发的路
径。如果存在一条,可以解释为$M$中以$s$开始不满足$\phi$的一条路径。所以,如果
没有这样的路径,则输出$yes,M,s\models \phi$。否则,如果存在这样的一条路径,输出
$no,M,s\models \phi$。在后一种情况中,可以从找到的路径中找出反例来表明不成立的
原因,这有助于调试系统。
LTL模型检测算法可以转化为CTL模型检测算法。NuSMV对LTL描述的
规范的验证就是通过转化为CTL模型检测算法实现的。



\section{本章小结}
本章主要介绍了协议形式化分析的基本概念和常用方法。介绍了安全协议的敌手模型以及常用分析工具,探讨了常用的协议分析技术,分析了每种技术的原理和在协议安全性分析中的应用,是本文后续研究的基础。
