%!TEX root = C:\Users\kobix\Documents\gradu\thesis1\thesiscc.tex
\chapter{总结与展望}

本文围绕安全协议形式化分析技术展开研究。针对递归协议安全性问题,提出了递归协议的安全性判定算法。针对符号方法中敌手描述能力不足的问题,拓展了传统Dolev-Yao敌手模型。针对安全协议实现状态机学习中的测试序列生成问题,提出了改进的测试序列生成算法。为了验证协议实现状态机的安全属性,提出了基于模型检测的协议实现分析技术。

\section{本文的主要研究工作}
本研究将形式化技术引入协议规范和协议实现的安全性分析中,以解决传统方法对安全协议分析中产生的敌手能力不足以及协议实现的逻辑漏洞难以分析等现实问题。
取得的主要成果如下:
%针对常规的符号执行方法无法处理多数据包输入的问题,从提高测试深度出发,提出了一种多数据包选择符号执行方法。

(1)针对传统形式化方法中敌手模型描述能力不足的问题,建立了一个基于操作语义敌手模型,该模型拓展了传统敌手模型,敌手将密码学上的敌手攻击分解为三个维度,可以通过三个维度组合建立对应的敌手模型。通过拓展敌手推导规则,增强敌手攻击能力,拓展了敌手攻击模型。传统的敌手模型中,协议参与者要么从开始就是诚实的并将一直保有自己的秘密,要么从开始就是完全的恶意用户并一直被敌手控制。这限制了协议中对诚实用户与被攻陷用户的情况分析。为此借鉴计算安全的敌手模型架构,为敌手能力进行拓展,建立了敌手长期密钥攻击模型和敌手短期数据攻击模型。拓展的敌手模型适用于协议强安全性分析。分析传统Dolev-Yao敌手模型中认为安全的协议,验证了在Dolev-Yao模型下安全的HMQV协议在拓展后的敌手模型下是不安全的。拓展的敌手模型能够对弱完美传输机密性、密钥泄露和本地状态泄露进行分析。利用该模型,对传统Dolev-Yao敌手模型中认为安全的HQMV协议进行分析,验证了在Dolev-Yao模型下安全的HMQV协议在拓展后的敌手模型下是不安全的。

(2)针对递归协议的递归原语难以形式化描述,导致递归协议安全性难以判定的问题,提出了基于拓展敌手模型的递归协议安全性判定算法,该算法能够对执行正常输入输出以及递归测试的协议进行验证,尤其是安全路由协议以及公钥证书路径的验证。通过对递归协议消息项、敌手能力建模,建立递归协议安全性判定约束系统。在拓展的敌手模型条件下,通过简化规则,将协议拓展敌手模型下的可推理项置于约束系统下。给出了基于映射的递归语言的判定过程,并证明这是一个NP问题。

(3)针对协议实现的分析模型难以提取问题,提出了基于状态机学习的安全协议实现分析技术。基于状态机的协议实现模型提取技术存在测试序列生成慢以及测测试序列生成树状态爆炸问题,针对测试序列生成慢的问题,提出了检查点算法,通过构造前缀树用来存储之前执行过的查询,使得测试序列生成所用步骤减少,减少了状态机生成时间。针对测试序列生成数量爆炸问题,提出了套接字约简方法,根据安全协议自身的安全机制,利用影响协议后续交互的套接字,约简相应生成的测试序列。提出了安全协议实现的状态机学习框架,使用改进算法生成测试序列,利用L*算法生成状态机模型,对TLS协议实现的状态机进行学习,成功学习到了GnuTLS,OpenSSL,NSS三种TLS协议实现的状态机模型。实验证明,与传统方法相比,该技术能够显著降低了学习过程的测试序列生成步骤、等价查询数量及所用学习时间。


(4)针对无法分析协议实现状态机的逻辑漏洞问题,提出了基于模型检测的协议实现分析技术通过对协议实现库的有限状态机提取,提取TLS协议实现库的状态机模型,建立协议安全属性模型,寻找协议实现中可能存在的异常行为,实现对协议实现库的自动化和系统化的分析。本论文对利用测试用例生成的协议实现库状态机进行安全属性建模,利用NuSMV工具,对提取的模型进行模型检测,实验证明本方法能够有效分析TLS协议实现库的状态机模型,找到协议实现库存在的逻辑漏洞及与规范不一致的缺陷。建立了基于模型检测的协议实现状态机模型分析框架。该框架以协议实现的状态机为输入,将协议规范及实现库声明的安全属性建模为线性时态逻辑,通过模型检测器NuSMV对安全属性进行验证,将验证结果分为属性满足和不满足。同时基于协议实现声明,建立了安全属性、密钥交换属性以及函数属性3类共12项安全属性,对验证过程中产生反例通过伪反例检查和反例重放两步组成的反例检查器进行筛选,确保该反例为真实执行。通过对TLS实现库的验证结果,找到了两种TLS协议实现的协议逻辑漏洞的真实执行,证明该方法所得的模型检测结果可以发现协议规范与协议实现之间的差异,有效发现协议实现状态机的逻辑漏洞。
%针对符号执行方法需要根据协议规范人工构造输入的问题,从提高自动化程度的角度出发(
%提出了一种基于模型的导向符号执行方法。该方法利用协议软件的交互特性和已知的协议规范信息,首先,对协议软件进行协议状态机模型推断,提出了一种针对协议软件的高效状态机学习算法。然后,利用协议模型对协议软件进行导向性符号执行,使得符号执行可以快速到达需要测试的协议状态,缓解了符号执行的状态空间爆炸问题,提高了协议软件安全性分析的测试效率。

%针对常规的符号执行方法无法联合测协议软件的接收端与发送端的问题,从发现互操作性方面的安全问题及缩减状态空间的角度出发(3)针对协议实现的分析模型难以提取问题,提出了基于状态机学习的安全协议实现分析技术。基于状态机的协议实现模型提取技术存在测试序列生成慢以及测测试序列生成树状态爆炸问题,针对测试序列生成慢的问题,提出了检查点算法,通过构造前缀树用来存储之前执行过的查询,使得测试序列生成所用步骤减少,减少了状态机生成时间。针对测试序列生成数量爆炸问题,提出了套接字约简方法,根据安全协议自身的安全机制,利用影响协议后续交互的套接字,约简相应生成的测试序列。提出了安全协议实现的状态机学习框架,使用改进算法生成测试序列,利用L*算法生成状态机模型,对TLS协议实现的状态机进行学习,成功学习到了GnuTLS,OpenSSL,NSS三种TLS协议实现的状态机模型。实验证明,与传统方法相比,该技术能够显著降低了学习过程的测试序列生成步骤、等价查询数量及所用学习时间。


%针对符号执行方法需要根据协议规范人工构造输入的问题,从提高自动化程度的角度出发(4)针对无法分析协议实现状态机的逻辑漏洞问题,提出了基于模型检测的协议实现分析技术通过对协议实现库的有限状态机提取,提取TLS协议实现库的状态机模型,建立协议安全属性模型,寻找协议实现中可能存在的异常行为,实现对协议实现库的自动化和系统化的分析。本文对利用测试用例生成的协议实现库状态机进行安全属性建模,利用NuSMV工具,对提取的模型进行模型检测,实验证明本方法能够有效分析TLS协议实现库的状态机模型,找到协议实现库存在的逻辑漏洞及与规范不一致的缺陷。建立了基于模型检测的协议实现状态机模型分析框架。该框架以协议实现的状态机为输入,将协议规范及实现库声明的安全属性建模为线性时态逻辑,通过模型检测器NuSMV对安全属性进行验证,将验证结果分为属性满足和不满足。同时基于协议实现声明,建立了安全属性、密钥交换属性以及函数属性3类共12项安全属性,对验证过程中产生反例通过伪反例检查和反例重放两步组成的反例检查器进行筛选,确保该反例为真实执行。通过对TLS实现库的验证结果,找到了两种TLS协议实现的协议逻辑漏洞的真实执行,证明该方法所得的模型检测结果可以发现协议规范与协议实现之间的差异,有效发现协议实现状态机的逻辑漏洞。


\section{未来工作展望}

虽然安全协议形式化分析技术已经在多个协议的验证中起到了关键作用,但是目前仍有部分问题值得进一步分析和研究。

(1)如何将符号方法与计算方法的优点进一步统一的问题。符号方法具有可自动化的优点,但由于抽象层次高,丢失了协议的执行细节,安全性证明结论不够充分;计算方法能够提供有力的安全证明,但十分复杂,不易自动化。如何找到能够提供强有力安全证明同时又能自动化验证的方法是极有研究价值的。

(2)如何对安全协议实现进行形式化分析的问题。安全协议实现作为安全协议的载体,既面临代码实现时出现的程序漏洞的风险,也面临着协议规范本身出错的风险。目前较为有效的方法是通过对协议实现提取其抽象模型进行验证,但是抽象不可避免会丢失执行细节,如何对具体执行的安全协议实现代码进行验证,是具有现实意义的。

(3)构造即正确的协议架构。目前大部分安全协议形式化分析的工作都集中在已发布的协议上,而一旦被分析到存在错误,就会面临安全协议重新设计的问题,这个过程的本质是找漏洞到修补漏洞到再找漏洞的循环。如果能够在协议设计阶段就通过严格的证明,通过精化的方式一步步构造正确的协议,是极具吸引力的研究工作。
